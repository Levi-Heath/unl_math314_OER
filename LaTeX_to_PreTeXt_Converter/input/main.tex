\documentclass{ximera}
%% You can put user macros here
%% However, you cannot make new environments

\listfiles

\graphicspath{
{./}
{./LTR-0070/}
{./VEC-0060/}
{./APP-0020/}
}

\usepackage{tikz}
\usepackage{tkz-euclide}
\usepackage{tikz-3dplot}
\usepackage{tikz-cd}
\usetikzlibrary{shapes.geometric}
\usetikzlibrary{arrows}
%\usetkzobj{all}
\pgfplotsset{compat=1.13} % prevents compile error.

%\renewcommand{\vec}[1]{\mathbf{#1}}
\renewcommand{\vec}{\mathbf}
\newcommand{\RR}{\mathbb{R}}
\newcommand{\dfn}{\textit}
\newcommand{\dotp}{\cdot}
\newcommand{\id}{\text{id}}
\newcommand\norm[1]{\left\lVert#1\right\rVert}
 
\newtheorem{general}{Generalization}
\newtheorem{initprob}{Exploration Problem}

\tikzstyle geometryDiagrams=[ultra thick,color=blue!50!black]

%\DefineVerbatimEnvironment{octave}{Verbatim}{numbers=left,frame=lines,label=Octave,labelposition=topline}



\usepackage{mathtools}


\title{QR Factorization} \license{CC BY-NC-SA 4.0}

\begin{document}

\begin{abstract}

\end{abstract}
\maketitle

\begin{onlineOnly}
\section*{QR Factorization}
\end{onlineOnly}

One of the main virtues of orthogonal
matrices is that they can be easily inverted---the transpose is the
inverse. This fact, combined with the factorization theorem in this
section, provides a useful way to simplify many matrix calculations (for
 example, in \href{https://ximera.osu.edu/oerlinalg/LinearAlgebra/RTH-0030/main}{least squares approximation}).


\begin{definition}\label{def:QR-factorization}
Let $A$ be an $m \times n$ matrix with independent columns. A \dfn{QR-factorization} of $A$ expresses it as $A = QR$ where $Q$ is $m \times n$ with orthonormal columns and $R$ is an invertible and upper triangular matrix with positive diagonal entries.
\end{definition}

The importance of the factorization
lies in the fact that there are computer algorithms that accomplish it
with good control over round-off error, making it particularly useful in
 matrix calculations. The factorization is a matrix version of the Gram-Schmidt process.


Suppose $A = \left[ \begin{array}{cccc}
|&|& &| \\
\vec{c}_{1} & \vec{c}_{2} & \cdots &  \vec{c}_{n}\\
|&|& &|
\end{array}\right]$ is an $m \times n$ matrix with linearly independent columns $\vec{c}_{1}, \vec{c}_{2}, \dots, \vec{c}_{n}$. The Gram-Schmidt algorithm can be applied to these columns to provide orthogonal columns $\vec{f}_{1}, \vec{f}_{2}, \dots, \vec{f}_{n}$ where $\vec{f}_{1} = \vec{c}_{1}$ and
\begin{equation*}
\vec{f}_{k} = \vec{c}_{k} - \frac{\vec{c}_{k} \dotp \vec{f}_{1}}{\norm{ \vec{f}_{1} }^2}\vec{f}_{1} + \frac{\vec{c}_{k} \dotp \vec{f}_{2}}{\norm{ \vec{f}_{2} }^2}\vec{f}_{2} - \dots - \frac{\vec{c}_{k} \dotp \vec{f}_{k-1}}{\norm{ \vec{f}_{k-1} }^2}\vec{f}_{k-1}
\end{equation*}
for each $k = 2, 3, \dots, n$. Now write $\vec{q}_{k} = \frac{1}{\norm{ \vec{f}_{k} }}\vec{f}_{k}$ for each $k$. Then $\vec{q}_{1}, \vec{q}_{2}, \dots, \vec{q}_{n}$ are orthonormal columns, and the above equation becomes
\begin{equation*}
\norm{ \vec{f}_{k} } \vec{q}_{k} = \vec{c}_{k} - (\vec{c}_{k} \dotp \vec{q}_{1})\vec{q}_{1} - (\vec{c}_{k} \dotp \vec{q}_{2})\vec{q}_{2} - \dots - (\vec{c}_{k} \dotp \vec{q}_{k-1})\vec{q}_{k-1}
\end{equation*}
Using these equations, express each $\vec{c}_{k}$ as a linear combination of the $\vec{q}_{i}$:
\begin{equation*}
\begin{array}{ccl}
\vec{c}_{1} &=& \norm{ \vec{f}_{1} } \vec{q}_{1} \\
\vec{c}_{2} &=& (\vec{c}_{2} \dotp \vec{q}_{1})\vec{q}_{1} + \norm{ \vec{f}_{2} } \vec{q}_{2} \\
\vec{c}_{3} &=& (\vec{c}_{3} \dotp \vec{q}_{1})\vec{q}_{1} + (\vec{c}_{3} \dotp \vec{q}_{2})\vec{q}_{2} + \norm{ \vec{f}_{3} } \vec{q}_{3} \\
\vdots && \vdots \\
\vec{c}_{n} &=& (\vec{c}_{n} \dotp \vec{q}_{1})\vec{q}_{1} + (\vec{c}_{n} \dotp \vec{q}_{2})\vec{q}_{2} + (\vec{c}_{n} \dotp \vec{q}_{3})\vec{q}_{3} + \dots + \norm{ \vec{f}_{n} } \vec{q}_{n}
\end{array}
\end{equation*}
These equations have a matrix form that gives the required factorization:
\begin{align}
A & = \left[ \begin{array}{ccccc}
|&|&|& &| \\
\vec{c}_{1} & \vec{c}_{2} & \vec{c}_{3} &\cdots &  \vec{c}_{n}\\
|&|&|& &|
\end{array}\right] \nonumber \\
&= \left[ \begin{array}{ccccc}
|&|&|& &| \\
\vec{q}_{1} & \vec{q}_{2} & \vec{q}_{3} & \cdots &  \vec{q}_{n}\\
|&|&|& &|
\end{array}\right] \left[ \begin{array}{ccccc}
\norm{ \vec{f}_{1} } & \vec{c}_{2} \dotp \vec{q}_{1} & \vec{c}_{3} \dotp \vec{q}_{1} & \cdots & \vec{c}_{n} \dotp \vec{q}_{1} \\
0 & \norm{ \vec{f}_{2} } & \vec{c}_{3} \dotp \vec{q}_{2} & \cdots & \vec{c}_{n} \dotp \vec{q}_{2} \\
0 & 0 & \norm{ \vec{f}_{3} } & \cdots & \vec{c}_{n} \dotp \vec{q}_{3} \\
\vdots & \vdots & \vdots & \ddots & \vdots \\
0 & 0 & 0 & \cdots & \norm{ \vec{f}_{n} }
\end{array} \right] \label{matrixFactEq}
\end{align}
Here the first factor $Q = \left[ \begin{array}{ccccc}
|&|&|& &| \\
\vec{q}_{1} & \vec{q}_{2} & \vec{q}_{3} & \cdots &  \vec{q}_{n}\\
|&|&|& &|
\end{array}\right]$ has orthonormal columns, and the second factor is an $n \times n$ upper triangular matrix $R$ with positive diagonal entries (and so is invertible). We record this in the following theorem.

\begin{theorem}[QR-Factorization]\label{th:QR-025133}
Every $m \times n$ matrix $A$ with linearly independent columns has a QR-factorization $A = QR$ where $Q$ has orthonormal columns and $R$ is upper triangular with positive diagonal entries.
\end{theorem}

The matrices $Q$ and $R$ in Theorem~\ref{th:QR-025133} are uniquely determined by $A$; we return to this below.

\begin{example}\label{ex:QR4x3-025139}
Find the QR-factorization of $A = \left[ \begin{array}{rrr}
1 & 1 & 0 \\
-1 & 0 & 1 \\
0 & 1 & 1 \\
0 & 0 & 1
\end{array}\right]$.

\begin{explanation}
  Denote the columns of $A$ as $\vec{c}_{1}$, $\vec{c}_{2}$, and $\vec{c}_{3}$, and observe that $\{\vec{c}_{1}, \vec{c}_{2}, \vec{c}_{3}\}$ is independent. If we apply the Gram-Schmidt algorithm to these columns, the result is:
\begin{equation*}
\vec{f}_{1} = \vec{c}_{1} = \left[ \begin{array}{r}
1  \\
-1  \\
0  \\
0
\end{array}\right], \quad \vec{f}_{2} = \vec{c}_{2} - \frac{1}{2}\vec{f}_{1} = \left[ \def\arraystretch{1.2} \begin{array}{r}
\frac{1}{2}  \\
\frac{1}{2}  \\
1  \\
0
\end{array}\right], \mbox{ and } \quad \vec{f}_{3} = \vec{c}_{3} + \frac{1}{2}\vec{f}_{1} - \vec{f}_{2} = \left[ \begin{array}{r}
0  \\
0  \\
0  \\
1
\end{array}\right].
\end{equation*}
Write $\vec{q}_{j} = \frac{1}{\norm{ \vec{f}_{j} }^2}\vec{f}_{j}$
 for each $j$, so $\{\vec{q}_{1}, \vec{q}_{2}, \vec{q}_{3}\}$ is orthonormal. Then equation (\ref{matrixFactEq}) preceding Theorem~\ref{th:QR-025133} gives $A = QR$ where
\begin{align*}
Q &= \left[ \begin{array}{ccc} | & | & | \\
\vec{q}_{1} & \vec{q}_{2} & \vec{q}_{3} \\
| & | & |
\end{array}\right] = \left[ \def\arraystretch{1.3}\begin{array}{ccc}
\frac{1}{\sqrt{2}} & \frac{1}{\sqrt{6}} & 0 \\
\frac{-1}{\sqrt{2}} & \frac{1}{\sqrt{6}} & 0 \\
0 & \frac{2}{\sqrt{6}} & 0 \\
0 & 0 & 1
\end{array}\right]  = \frac{1}{\sqrt{6}} \left[ \begin{array}{ccc}
\sqrt{3} & 1 & 0 \\
-\sqrt{3} & 1 & 0 \\
0 & 2 & 0 \\
0 & 0 & \sqrt{6}
\end{array} \right]
\\
R &= \left[ \begin{array}{ccc}
\norm{ \vec{f}_{1} } & \vec{c}_{2} \dotp \vec{q}_{1} & \vec{c}_{3} \dotp \vec{q}_{1} \\
0 & \norm{ \vec{f}_{2} } & \vec{c}_{3} \dotp \vec{q}_{2} \\
0 & 0 & \norm{ \vec{f}_{3} } \\
\end{array} \right] = \left[ \def\arraystretch{1.5} \begin{array}{ccc}
\sqrt{2} & \frac{1}{\sqrt{2}} & \frac{-1}{\sqrt{2}} \\
0 & \frac{\sqrt{3}}{\sqrt{2}} & \frac{\sqrt{3}}{\sqrt{2}} \\
0 & 0 & 1
\end{array} \right] = \frac{1}{\sqrt{2}}\left[ \begin{array}{ccc}
2 & 1 & -1 \\
0 & \sqrt{3} & \sqrt{3} \\
0 & 0 & \sqrt{2}
\end{array} \right]
\end{align*}

The reader can verify that indeed $A = QR$.
\end{explanation}
\end{example}

If a matrix $A$ has independent rows and we apply QR-factorization to $A^{T}$, the result is:

\begin{corollary}\label{cor:QR-transpose-025162}
If $A$ has independent rows, then $A$ factors uniquely as $A = LP$ where $P$ has orthonormal rows and $L$ is an invertible lower triangular matrix with positive main diagonal entries.
\end{corollary}

Since a square matrix with orthonormal columns is orthogonal, we have

\begin{theorem}\label{th:025166}
Every square invertible matrix $A$ has factorizations $A = QR$ and $A = LP$ where $Q$ and $P$ are orthogonal, $R$ is upper triangular with positive diagonal entries, and $L$ is lower triangular with positive diagonal entries.
\end{theorem}

We now take the time to prove the uniqueness of the QR-factorization.

\begin{theorem}\label{th:QR-unique-025187}
Let $A$ be an $m \times n$ matrix with independent columns. If $A = QR$ and $A = Q_{1}R_{1}$ are QR-factorizations of $A$, then $Q_{1} = Q$ and $R_{1} = R$.
\end{theorem}

\begin{proof}
Write $Q = \left[ \begin{array}{cccc}
|&|& &| \\
\vec{c}_{1} & \vec{c}_{2} & \cdots &  \vec{c}_{n}\\
|&|& &|
\end{array}\right]$  and $Q_{1} =  \left[ \begin{array}{cccc}
|&|& &| \\
\vec{d}_{1} & \vec{d}_{2} & \cdots &  \vec{d}_{n}\\
|&|& &|
\end{array}\right]$ in terms of their columns, and observe first that $Q^TQ = I_{n} = Q_{1}^TQ_{1}$ because $Q$ and $Q_{1}$ have orthonormal columns. Hence it suffices to show that $Q_{1} = Q$ (then $R_{1} = Q_{1}^TA = Q^TA = R$). Since $Q_{1}^TQ_{1} = I_{n}$, the equation $QR = Q_{1}R_{1}$ gives $Q_{1}^TQ = R_{1}R^{-1}$; for convenience we write this matrix as
\begin{equation*}
Q_{1}^TQ = R_{1}R^{-1} = \left[ \begin{array}{c} t_{ij} \end{array}\right]
\end{equation*}
This matrix is upper triangular with positive diagonal elements (since this is true for $R$ and $R_{1}$), so $t_{ii} > 0$ for each $i$ and $t_{ij} = 0$ if $i > j$. On the other hand, the $(i, j)$-entry of $Q_{1}^TQ$ is $\vec{d}_{i}^T\vec{c}_{j} = \vec{d}_{i} \dotp \vec{c}_{j}$, so we have $\vec{d}_{i} \dotp \vec{c}_{j} = t_{ij}$ for all $i$ and $j$. But each $\vec{c}_{j}$ is in $\mbox{span}\{\vec{d}_{1}, \vec{d}_{2}, \dots, \vec{d}_{n}\}$ because $Q = Q_{1}(R_{1}R^{-1})$. We know how to write a vector as a linear combination of an orthonormal basis (using Corollary~\ref{cor:orthonormal} from \href{https://ximera.osu.edu/oerlinalg/LinearAlgebra/RTH-0010/main}{Orthogonality and Projections}):
\begin{equation*}
\vec{c}_{j} = (\vec{d}_{1} \dotp \vec{c}_{j})\vec{d}_{1} + (\vec{d}_{2} \dotp \vec{c}_{j})\vec{d}_{2} + \dots + (\vec{d}_{n} \dotp \vec{c}_{j})\vec{d}_{n} = t_{1j}\vec{d}_{1} + t_{2j}\vec{d}_{2} + \dots + t_{jj}\vec{d}_{i},
\end{equation*}
because $\vec{d}_{i} \dotp \vec{c}_{j} = t_{ij} = 0$ if $i > j$. The first few equations here are
\begin{equation*}
\begin{array}{ccl}
\vec{c}_{1} &=& t_{11}\vec{d}_{1} \\
\vec{c}_{2} &=& t_{12}\vec{d}_{1} + t_{22}\vec{d}_{2} \\
\vec{c}_{3} &=& t_{13}\vec{d}_{1} + t_{23}\vec{d}_{2} + t_{33}\vec{d}_{3} \\
\vec{c}_{4} &=& t_{14}\vec{d}_{1} + t_{24}\vec{d}_{2} + t_{34}\vec{d}_{3} + t_{44}\vec{d}_{4} \\
\vdots && \vdots
\end{array}
\end{equation*}
The first of these equations gives $1 = \norm{ \vec{c}_{1} } = \norm{ t_{11}\vec{d}_{1} } = | t_{11} | \norm{ \vec{d}_{1} } = t_{11}$, whence $\vec{c}_{1} = \vec{d}_{1}$. But then we have $t_{12} = \vec{d}_{1} \dotp \vec{c}_{2} = \vec{c}_{1} \dotp \vec{c}_{2} = 0$, so the second equation becomes $\vec{c}_{2} = t_{22}\vec{d}_{2}$. Now a similar argument gives $\vec{c}_{2} = \vec{d}_{2}$, and then $t_{13} = 0$ and $t_{23} = 0$ follows in the same way. Hence $\vec{c}_{3} = t_{33}\vec{d}_{3}$ and $\vec{c}_{3} = \vec{d}_{3}$. Continue in this way to get $\vec{c}_{i} = \vec{d}_{i}$ for all $i$. This proves that $Q_{1} = Q$.
\end{proof}

\section*{QR-Algorithm for approximating eigenvalues}\label{sec:QRalgorithm}

In \href{}{The Power Method and the Dominant Eigenvalue}, we learned about an iterative method for computing eigenvalues.  We also mentioned that a better method for approximating the eigenvalues of an invertible matrix $A$ depends on the QR-factorization of $A$.  While it is beyond the scope of this book to pursue a detailed discussion of this method, we give an example and conclude with some remarks on the QR-algorithm. The interested reader is referred to
 J. M. Wilkinson, \textit{The Algebraic Eigenvalue Problem} (Oxford, England: Oxford University Press, 1965)\index{\textit{The Algebraic Eigenvalue Problem} (Wilkinson)} or G. W. Stewart, \textit{Introduction to Matrix Computations} (New York: Academic Press, 1973)\index{\textit{Introduction to Matrix Computations} (Stewart)}.

The \dfn{QR-algorithm} uses QR-factorization repeatedly to create a sequence of matrices $A_{1} =A, A_{2}, A_{3}, \dots,$ as follows:

\begin{enumerate}
\item Define $A_{1} = A$ and factor it as $A_{1} = Q_{1}R_{1}$.

\item Define $A_{2} = R_{1}Q_{1}$ and factor it as $A_{2} = Q_{2}R_{2}$.

\item Define $A_{3} = R_{2}Q_{2}$ and factor it as $A_{3} = Q_{3}R_{3}$.
\\ \hspace*{4em} $\vdots$


\end{enumerate}

In general, $A_{k}$ is factored as $A_{k} = Q_{k}R_{k}$ and we define $A_{k + 1} = R_{k}Q_{k}$. Then $A_{k + 1}$ is similar to $A_{k}$ [in fact, $A_{k+1} = R_{k}Q_{k} = (Q_{k}^{-1}A_{k})Q_{k}$], and hence each $A_{k}$ has the same eigenvalues as $A$. If the eigenvalues of $A$ are real and have distinct absolute values, the remarkable thing is that the sequence of matrices $A_{1}, A_{2}, A_{3}, \dots$ converges to an upper triangular matrix with these eigenvalues on the main diagonal. [See below for the case of complex eigenvalues.]

\begin{example}\label{QR-algortihm-2x2-025425}
If $A = \left[ \begin{array}{rr}
1 & 1 \\
2 & 0
\end{array}\right]$ as in Example~\ref{ex:2x2PowerMethod025326} of \href{}{The Power Method and the Dominant Eigenvalue}, use the QR-algorithm to approximate the eigenvalues.

\begin{explanation}
  The matrices $A_{1}$, $A_{2}$, and $A_{3}$ are as follows:
\begin{align*}
A_{1} = & \left[ \begin{array}{rr}
1 & 1 \\
2 & 0
\end{array}\right] = Q_{1}R_{1} \quad \mbox{ where } Q_{1} = \frac{1}{\sqrt{5}}\left[ \begin{array}{rr}
1 & 2 \\
2 & -1
\end{array}\right] \mbox{ and } R_{1} =  \frac{1}{\sqrt{5}}\left[ \begin{array}{rr}
5 & 1 \\
0 & 2
\end{array}\right] \\
A_{2} = & \frac{1}{5}\left[ \begin{array}{rr}
7 & 9 \\
4 & -2
\end{array}\right] = \left[ \begin{array}{rr}
1.4 & -1.8 \\
-0.8 & -0.4
\end{array}\right]= Q_{2}R_{2} \\
&\mbox{ where } Q_{2} = \frac{1}{\sqrt{65}}\left[ \begin{array}{rr}
7 & 4 \\
4 & -7
\end{array}\right] \mbox{ and } R_{2} =  \frac{1}{\sqrt{65}}\left[ \begin{array}{rr}
13 & 11 \\
0 & 10
\end{array}\right] \\
A_{3} = &\frac{1}{13}\left[ \begin{array}{rr}
27 & -5 \\
8 & -14
\end{array}\right] = \left[ \begin{array}{rr}
2.08 & -0.38 \\
0.62 & -1.08
\end{array}\right]
\end{align*}
This is converging to $\left[ \begin{array}{rr}
2 & \ast \\
0 & -1
\end{array}\right]$ and so is approximating the eigenvalues $2$ and $-1$ on the main diagonal.
\end{explanation}
\end{example}

\subsubsection*{Shifting.} We first learned about the concept of \dfn{shifting} in \href{}{The Power Method and the Dominant Eigenvalue}.  Convergence is accelerated if, at stage $k$ of the algorithm, a number $\tau_{k}$ is chosen and $A_{k} - \tau_{k}I$ is factored in the form $Q_{k}R_{k}$ rather than $A_{k}$ itself. Then
\begin{equation*}
Q_{k}^{-1}A_{k}Q_{k} = Q_{k}^{-1}(Q_{k}R_{k} + \tau_{k}I)Q_{k} = R_{k}Q_{k} + \tau_{k}I
\end{equation*}
so we take $A_{k+1} = R_{k}Q_{k} + \tau_{k}I$. If the shifts $\tau_{k}$ are carefully chosen, convergence can be greatly improved.

\subsubsection*{Preliminary Preparation.} A matrix such as
\begin{equation*}
\left[ \begin{array}{rrrrr}
\ast  & \ast & \ast & \ast & \ast  \\
\ast  & \ast & \ast & \ast & \ast  \\
0  & \ast & \ast & \ast & \ast  \\
0  & 0 & \ast & \ast & \ast  \\
0  & 0 & 0 & \ast & \ast  \\
\end{array}\right]
\end{equation*}
is said to be in \dfn{upper Hessenberg} form, and the QR-factorizations of such matrices are greatly simplified. Given an $n \times n$ matrix $A$, a series of orthogonal matrices $H_{1}, H_{2}, \dots, H_{m}$ (called \dfn{Householder matrices} can be easily constructed such that
\begin{equation*}
B = H_{m}^T \cdots H_{1}^TAH_{1} \cdots H_{m}
\end{equation*}
is in upper Hessenberg form. Then the QR-algorithm can be efficiently applied to $B$ and, because $B$ is similar to $A$, it produces the eigenvalues of $A$.

\subsubsection*{Complex Eigenvalues.} If some of the eigenvalues of a real matrix $A$ are not real, the QR-algorithm converges to a block upper triangular matrix where the diagonal blocks are either $1 \times 1$ (the real eigenvalues) or $2 \times 2$ (each providing a pair of conjugate complex eigenvalues of $A$).

\section*{Practice Problems}

\begin{problem}\label{prob:findQR}
In each case find the QR-factorization of $A$.

\begin{enumerate}
\item $A = \left[ \begin{array}{rr}
1 & -1 \\
-1 & 0
\end{array}\right]$
\item $A = \left[ \begin{array}{rr}
2 & 1 \\
1 &1
\end{array}\right]$
\item $A = \left[ \begin{array}{rrr}
1 & 1 & 1 \\
1 & 1 & 0 \\
1 & 0 & 0 \\
0 & 0 & 0
\end{array}\right]$
\item $A = \left[ \begin{array}{rrr}
1 & 1 & 0 \\
-1 & 0 & 1 \\
0 & 1 & 1 \\
1 & -1 & 0
\end{array}\right]$
\end{enumerate}
Click the arrow to see answer.
\begin{expandable}
\begin{enumerate} 
\item  $Q = \frac{1}{\sqrt{5}}\left[ \begin{array}{rr}
2 & -1 \\
1 & 2
\end{array}\right]$,
 $R = \frac{1}{\sqrt{5}}\left[ \begin{array}{rr}
5 & 3 \\
0 & 1
\end{array}\right]$

\item  $Q = \frac{1}{\sqrt{3}}\left[ \begin{array}{rrr}
1 & 1 & 0 \\
-1 & 0 & 1 \\
0 & 1 & 1 \\
1 & -1 & 1
\end{array}\right]$, \\
$R = \frac{1}{\sqrt{3}}\left[ \begin{array}{rrr}
3 & 0 & -1 \\
0 & 3 & 1 \\
0 & 0 & 2
\end{array}\right]$

\end{enumerate}
\end{expandable}
\end{problem}

%\begin{problem}\label{prob:indcolumns}
%Let $A$ and $B$ denote matrices.


%\begin{enumerate} 
%\item If $A$ and $B$ have independent columns, show that $AB$ has independent columns. %[\textit{Hint}: Theorem~\ref{thm:015672}.]

%\item Show that $A$ has a QR-factorization if and only if $A$ has independent columns.
%\begin{hint}
%If $A$ has a QR-factorization, use (a). For the converse use Theorem~\ref{thm:025133}.
%\end{hint}

%\item If $AB$ has a QR-factorization, show that the same is true of $B$ but not necessarily $A$.
%\end{enumerate}
%\begin{hint}
%Consider $AA^{T}$ where $A = \left[ \begin{array}{rrr}
%1 & 0 & 0 \\
%1 & 1 & 1
%\end{array}\right]$.
%\end{hint}

%\end{problem}

\begin{problem}\label{prob:take_diag_positive}
If $R$ is upper triangular and invertible, show that there exists a diagonal matrix $D$ with diagonal entries $\pm 1$ such that $R_{1} = DR$ is invertible, upper triangular, and has positive diagonal entries.
\end{problem}

\begin{problem}\label{prob:fullQR}
If $A$ has independent columns, let \\ $A = QR$ where $Q$ has orthonormal columns and $R$ is invertible and upper triangular. (Some authors do not require a $QR$-factorization to have positive diagonal entries.) Show that there is a diagonal matrix $D$ with diagonal entries $\pm 1$ such that $A = (QD)(DR)$ is the QR-factorization of $A$.
\begin{hint}
See Practice Problem \ref{prob:take_diag_positive}
\end{hint}
\end{problem}

\begin{problem}
In each case, find the exact eigenvalues and then approximate them using the QR-algorithm.


\begin{enumerate}
\item $A = \left[ \begin{array}{rr}
1 & 1 \\
1 & 0
\end{array}\right]$
\item $A = \left[ \begin{array}{rr}
3 & 1 \\
1 & 0
\end{array}\right]$
\end{enumerate}

Click the arrow to see answer.
\begin{expandable}
\begin{enumerate} 
\item  Eigenvalues $\lambda_{1} = \frac{1}{2}(3 + \sqrt{13}) = 3.302776$, $\lambda_{2} = \frac{1}{2}(3 - \sqrt{13}) = -0.302776$

$A_{1} = \left[ \begin{array}{rr}
3 & 1 \\
1 & 0
\end{array}\right]$, $Q_{1} = \frac{1}{\sqrt{10}}\left[ \begin{array}{rr}
3 & -1 \\
1 & 3
\end{array}\right]$, $R_{1} = \frac{1}{\sqrt{10}}\left[ \begin{array}{rr}
10 & 3 \\
0 & -1
\end{array}\right]$ \\
$A_{2} = \frac{1}{10}\left[ \begin{array}{rr}
33 & -1 \\
-1 & -3
\end{array}\right]$, \\ $Q_{2} = \frac{1}{\sqrt{1090}}\left[ \begin{array}{rr}
33 & 1 \\
-1 & 33
\end{array}\right]$, \\ $R_{2} = \frac{1}{\sqrt{1090}}\left[ \begin{array}{rr}
109 & -3 \\
0 & -10
\end{array}\right]$ \\
$A_{3} = \frac{1}{109}\left[ \begin{array}{rr}
360 & 1 \\
1 & -33
\end{array}\right]$ \\ ${} = \left[ \begin{array}{rr}
3.302775 & 0.009174 \\
0.009174 & -0.302775
\end{array}\right]$

\end{enumerate}
\end{expandable}
\end{problem}

\begin{problem}\label{prob:QR-symmetric}
If $A$ is symmetric, show that each matrix $A_{k}$ in the QR-algorithm is also symmetric. Deduce that they converge to a diagonal matrix.

Click the arrow to see answer.
\begin{expandable}
Use induction on $k$. If $k = 1$, $A_{1} = A$. In general $A_{k+1} = Q_{k}^{-1}A_{k}Q_{k} = Q_{k}^{T}A_{k}Q_{k}$, so the fact that $A_{k}^{T} = A_{k}$ implies $A_{k+1}^{T} = A_{k+1}$. The eigenvalues of $A$ are all real (Theorem \ref{cor:ews_symmetric_real}), so the $A_{k}$ converge to an upper triangular matrix $T$. But $T$ must also be symmetric (it is the limit of symmetric matrices), so it is diagonal.
\end{expandable}
\end{problem}

\begin{problem}\label{QR-special-2x2}
Apply the QR-algorithm to \\ $A = \left[ \begin{array}{rr}
2 & -3 \\
1 & -2
\end{array}\right]$. Explain.
\end{problem}

\begin{problem}\label{prob:analyzeQRalgorithm}
Given a matrix $A$, let $A_{k}$, $Q_{k}$, and $R_{k}$, $k \geq 1$, be the matrices constructed in the QR-algorithm. Show that $A_{k} = (Q_{1}Q_{2} \cdots Q_{k})(R_{k} \cdots R_{2}R_{1})$ for each $k \geq 1$ and hence that this is a QR-factorization of $A_{k}$. 
\begin{hint}
Show that $Q_{k}R_{k} = R_{k-1}Q_{k-1}$ for each $k \geq 2$, and use this equality to compute $(Q_{1}Q_{2} \cdots Q_{k})(R_{k} \cdots R_{2}R_{1})$ ``from the centre out.'' Use the fact that $(AB)^{n+1} = A(BA)^{n}B$ for any square matrices $A$ and $B$.
\end{hint}
\end{problem}

\section*{Text Source} This section was adapted from Sections 8.4 and 8.5 of Keith Nicholson's \href{https://open.umn.edu/opentextbooks/textbooks/linear-algebra-with-applications}{\it Linear Algebra with Applications}. (CC-BY-NC-SA)

W. Keith Nicholson, {\it Linear Algebra with Applications}, Lyryx 2018, Open Edition, pp. 437--445.
\end{document}