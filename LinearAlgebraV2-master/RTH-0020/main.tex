\documentclass{ximera}
%% You can put user macros here
%% However, you cannot make new environments

\listfiles

\graphicspath{
{./}
{./LTR-0070/}
{./VEC-0060/}
{./APP-0020/}
}

\usepackage{tikz}
\usepackage{tkz-euclide}
\usepackage{tikz-3dplot}
\usepackage{tikz-cd}
\usetikzlibrary{shapes.geometric}
\usetikzlibrary{arrows}
%\usetkzobj{all}
\pgfplotsset{compat=1.13} % prevents compile error.

%\renewcommand{\vec}[1]{\mathbf{#1}}
\renewcommand{\vec}{\mathbf}
\newcommand{\RR}{\mathbb{R}}
\newcommand{\dfn}{\textit}
\newcommand{\dotp}{\cdot}
\newcommand{\id}{\text{id}}
\newcommand\norm[1]{\left\lVert#1\right\rVert}
 
\newtheorem{general}{Generalization}
\newtheorem{initprob}{Exploration Problem}

\tikzstyle geometryDiagrams=[ultra thick,color=blue!50!black]

%\DefineVerbatimEnvironment{octave}{Verbatim}{numbers=left,frame=lines,label=Octave,labelposition=topline}



\usepackage{mathtools}

 \title{Orthogonal Complements and Decompositions} \license{CC BY-NC-SA 4.0}

\begin{document}

\begin{abstract}
\end{abstract}
\maketitle

\begin{onlineOnly}
\section*{Orthogonal Complements and Decompositions}
\end{onlineOnly}

\subsection*{Orthogonal Complements}
We will now consider the set of vectors that are orthogonal to every vector in a given subspace.  As a quick example, consider the $xy$-plane in $\RR^3$.  Clearly, every scalar multiple of the standard unit vector $\vec{k}$ in $\RR^3$ is orthogonal to every vector in the $xy$-plane.  We say that the set $\{c\vec{k} \mid c \in\RR\}$ is an \dfn{orthogonal complement} of $\{a\vec{i}+b\vec{j} \mid a, b \in\RR\}$.

  \begin{definition}[Orthogonal Complement of a Subspace of $\RR^n$]\label{def:023776}
If $W$ is a subspace of $\RR^n$, define the \dfn{orthogonal complement} $W^\perp$ of $W$ (pronounced ``$W$-perp'') by
\begin{equation*}
W^\perp = \{\vec{x} \in\RR^n \mid \vec{x} \dotp \vec{y} = 0 \mbox{ for all } \vec{y} \in W\}
\end{equation*}
\end{definition}

\begin{center}
\tdplotsetmaincoords{70}{130}
	\begin{tikzpicture}[scale=1]
\filldraw[blue, opacity=0.3] (2,0,2)--(2,0,-2)--(-2,0,-2)--(-2,0,2)--cycle;
    \draw[->,line width=0.4mm, -stealth, blue](0,-2,0)--(0,3,0) ;%normal to grey
 \node[label={below right:$W$}] at (0.8,0.1,-2) {};
    \node[label={below right:$W^\perp$}] at (0,3,0) {};   
   \node[fill,circle,inner sep=1.5pt] at (0,0,0) {};
     \end{tikzpicture}
\end{center}

The following theorem collects some
useful properties of the orthogonal complement; the proof of \ref{th:023783a} and \ref{th:023783b}
 is left as Practice Problem \ref{prob:8_1_6}.

\begin{theorem}\label{th:023783}
Let $W$ be a subspace of $\RR^n$.
\begin{enumerate}
\item\label{th:023783a} $W^\perp$ is a subspace of $\RR^n$.

\item\label{th:023783b} $\{\vec{0}\}^\perp = \RR^n$ and $(\RR^n)^\perp = \{\vec{0}\}$.

\item\label{th:023783c} If $W = \mbox{span}\left(\vec{x}_{1}, \vec{x}_{2}, \dots, \vec{x}_{k}\right)$ then $W^\perp = \{\vec{x} \mbox{ in } \RR^n \mid \vec{x} \dotp \vec{x}_{i} = 0 \mbox{ for } i = 1, 2, \dots, k\}$.

\end{enumerate}
\end{theorem}

\begin{proof}[Proof of Part~\ref{th:023783c}:]
We must show that $W^\perp = \{\vec{x} \mid \vec{x} \dotp \vec{x}_{i} = 0 \mbox{ for each } i\}$.  To show that two sets are equal, we must show that all elements of one set are included in the other set, and then we must show the reverse inclusion.
%To show that two sets $\mathcal{S}$ and $\mathcal{T}$ are equal, a common strategy is to show that $\mathcal{S} \subseteq \mathcal{T}$ and also $\mathcal{T} \subseteq \mathcal{S}$.  We employ this strategy now.

If $\vec{x}$ is in $W^\perp$ then $\vec{x} \dotp \vec{x}_{i} = 0$ for all $i$ because each $\vec{x}_{i}$ is in $W$. This shows $W^\perp \subseteq \{\vec{x} \mid \vec{x} \dotp \vec{x}_{i} = 0 \mbox{ for each } i\}$. For the reverse inclusion, suppose that $\vec{x} \dotp \vec{x}_{i} = 0$ for all $i$; we need to show that $\vec{x}$ is in $W^\perp$.  We need to show $\vec{x} \dotp \vec{y} = 0$ for each $\vec{y}$ in $W$. We can write $\vec{y} = c_{1}\vec{x}_{1} + c_{2}\vec{x}_{2} + \dots  + c_{k}\vec{x}_{k}$, where each $c_{i}$ is in $\RR$. Then
\begin{equation*}
\vec{x} \dotp \vec{y} = c_{1}(\vec{x} \dotp \vec{x}_{1}) + c_{2}(\vec{x} \dotp \vec{x}_{2})+ \dots +c_{k}(\vec{x} \dotp \vec{x}_{k}) = c_{1}0 + c_{2}0 + \dots + c_{k}0 = 0
\end{equation*}
 as required, and the proof of equality is complete.

\end{proof}

\begin{example}\label{ex:023829}
Find a basis for $W^\perp$ if $W = \mbox{span}\left(\begin{bmatrix}
  1 \\ -1 \\ 2 \\ 0
  \end{bmatrix},
  \begin{bmatrix}
  1 \\ 0 \\ -2 \\ 3
  \end{bmatrix}\right)$ in $\RR^4$.

\begin{explanation}
  By Theorem~\ref{th:023783}, $\vec{x} = \begin{bmatrix}
  x \\ y \\ z \\ w
  \end{bmatrix}$
  is in $W^\perp$ if and only if $\vec{x}$ is orthogonal to both 
  $\vec{v}_1 = \begin{bmatrix}
  1 \\ -1 \\ 2 \\ 0
  \end{bmatrix}$ and
  $\vec{v}_2 = \begin{bmatrix}
  1 \\ 0 \\ -2 \\ 3
  \end{bmatrix}$; that is,
  $\vec{x} \dotp \vec{v}_1 = 0$ and $\vec{x} \dotp \vec{v}_2 = 0$, or
\begin{equation*}
\begin{array}{rrrrrrrr}
x & - & y & + & 2z & & & =0\\
x & & & - & 2z & +& 3w & =0
\end{array}
\end{equation*}
Using Gaussian elimination on this system gives $W^\perp = \mbox{span}\left(\begin{bmatrix}
  2 \\ 4 \\ 1 \\ 0
  \end{bmatrix},
  \begin{bmatrix}
  3 \\ 3 \\ 0 \\ -1
  \end{bmatrix}\right)$.  You are asked to confirm this in Practice Problem \ref{prob:Uperp} (which serves as a wonderful review of concepts we covered earlier in the course!).  
\end{explanation}
\end{example}

Some of the important subspaces we studied earlier are orthogonal complements of each other.  Recall the following definitions associated with an $m \times n$ matrix $A$.
\begin{enumerate}
    \item The \dfn{null space} of $A$, $\mbox{null}(A) = \{\vec{x}\in \RR^n \mid A\vec{x} = \vec{0}\}$, is a subspace of $\RR^n$.
    \item The \dfn{row space} of $A$, $\mbox{row}(A) = \mbox{span} ( \mbox{the rows of } A)$, is a subspace of $\RR^n$.
    \item The \dfn{column space} of $A$, $\mbox{col}(A) = \mbox{span} ( \mbox{the columns of } A)$, is a subspace of $\RR^m$.
\end{enumerate}

\begin{exploration}\label{exp:discoverortho}
In the following GeoGebra interactive, you can change the coordinates of the vectors $\vec{v}$ and $\vec{w}$ using the sliders.  (At this stage make sure that $\vec{v}$ and $\vec{w}$ are not collinear.) Let $A=\begin{bmatrix}-&\vec{v}&-\\-&\vec{w}&-\end{bmatrix}$.  Then $\mbox{row}(A) = \mbox{span}(\vec{v},\vec{w})$.  RIGHT-CLICK and DRAG to rotate the coordinate system for a better view.

\pdfOnly{
Access GeoGebra interactives through the online version of this text at 

\href{https://ximera.osu.edu/oerlinalg}{https://ximera.osu.edu/oerlinalg}.
}

\begin{onlineOnly}   
\begin{center}
\geogebra{f6eavqxs}{950}{800}
\end{center}
\end{onlineOnly}

\begin{enumerate}
        \item Follow the prompts in the interactive to visualize $\mbox{row}(A)$ and $\mbox{null}(A)$.  What relationships do you observe between $\mbox{row}(A)$ and $\mbox{null}(A)$? 
       \item  It is possible to ``break" this interactive (for certain choices of $\vec{v}$ and $\vec{w}$). If $\vec{v}$ and $\vec{w}$ are scalar multiples of each other, then $\mbox{row}(A)$ is a \wordChoice{\choice{Point}, \choice[correct]{Line},\choice{Plane}}, and the dimension of $\mbox{null}(A)$ is \wordChoice{\choice{1}, \choice[correct]{2},\choice{3}}.  The interactive does not accommodate this situation.  To see what happens when $\vec{v}$ and $\vec{w}$ are scalar multiples of each other, see Practice Problem \ref{prob:brokenInteractive}.
        \end{enumerate}
\end{exploration}


\begin{theorem}\label{th:4subspaces}
Let $A$ be an $m \times n$ matrix.  Then we have:
\begin{enumerate}
\item\label{th:4subspacesa} $\mbox{null}(A) = (\mbox{row}(A))^\perp$;
\item\label{th:4subspacesb} $\mbox{null}(A^T) = (\mbox{col}(A))^\perp$.
\end{enumerate}
\end{theorem}

Before proving this theorem, let's examine what it says about a couple of our examples.  In Example \ref{ex:023829}, we solved for the unknown vectors $\vec{x} = \begin{bmatrix}
  x \\ y \\ z \\ w
  \end{bmatrix}$. Notice that this is equivalent to creating a $2 \times 4$ matrix $A$ whose rows are $\vec{v}_1$ and $\vec{v}_2$, and then finding the null space of that matrix $A$.  You can check that a basis for $\mbox{null}\left(\begin{bmatrix}
  1 & -1 & 2 & 0 \\
  1 & 0 & -2 & 3
  \end{bmatrix}\right)$
is given by $\left\{\begin{bmatrix}
  2 \\ 4 \\ 1 \\ 0
  \end{bmatrix},
  \begin{bmatrix}
  3 \\ 3 \\ 0 \\ -1
  \end{bmatrix}\right\}$.

\begin{example}\label{ex:4subspaces}
Let
$$A=\begin{bmatrix}2&-1&1&-4&1\\1&0&3&3&0\\-2&1&-1&5&2\\4&-1&7&2&1\end{bmatrix}$$
Verify each of the statements in Theorem~\ref{th:4subspaces}.

\begin{explanation}
 We compute $\mbox{rref}(A)$ to find a basis for   $\mbox{null}(A)$, $\mbox{row}(A)$, and $\mbox{col}(A)$.  After some work we arrive at:
 $\mbox{null}(A) = \mbox{span}\left(\begin{bmatrix}-3\\-5\\1\\0\\0\end{bmatrix}, \begin{bmatrix}9\\31\\0\\-3\\1\end{bmatrix}\right)$ 
 and
 $$\mbox{row}(A)=\mbox{span}\Big(\begin{bmatrix}1&0&3&0&-9\end{bmatrix},
\begin{bmatrix}0&1&5&0&-31\end{bmatrix},
\begin{bmatrix}0&0&0&1&3\end{bmatrix}\Big).$$  (See examples in \link[Subspaces of $\RR^n$ Associated with Matrices]{https://ximera.osu.edu/oerlinalg/LinearAlgebra/VSP-0040/main} for the details.) It is easy to check that each of the basis vectors of $\mbox{null}(A)$ is orthogonal to each of the basis vectors of $\mbox{row}(A)$, demonstrating the first part of Theorem~\ref{th:4subspaces}.  You will be asked to demonstrate the second part of Theorem~\ref{th:4subspaces} for this example in Practice Problem \ref{prob:finishex4subspaces}.
\end{explanation}
\end{example}

We now return to the proof of Theorem \ref{th:4subspaces}.

\begin{proof}[Proof of Theorem~\ref{th:4subspaces}:]
Let $\vec{x}\in\RR^n$.  $\vec{x}\in\left(\mbox{row}(A)\right)^\perp$ if and only if x is orthogonal to every row of $A$.  But this is true if and only if $A\vec{x}=\vec{0}$, which is equivalent to saying $\vec{x}\in\mbox{null}(A)$, which proves \ref{th:4subspacesa}.  To prove \ref{th:4subspacesb}, we simply replace $A$ with $A^T$, and we may apply \ref{th:4subspacesa} since $\mbox{col}(A) = \mbox{row}(A^T)$.
\end{proof}

\subsection*{Orthogonal Decomposition Theorem}

Now that we have defined the orthogonal complement of a subspace, we are ready to state the main theorem of this section.  If you have studied physics or multi-variable calculus, you are familiar with the idea of expressing a vector in as the sum of its tangential and normal components. (If you haven't yet taken those courses, this section will help to prepare you for them!)  The following theorem is a generalization of this idea.

\begin{theorem}[Orthogonal Decomposition Theorem]\label{th:OrthoDecomp}
Let $W$ be a subspace of $\RR^n$ and let $\vec{x} \in \RR^n$.  Then there exist unique vectors $\vec{w} \in W$ and $\vec{w}^\perp \in W^\perp$ such that $\vec{x} = \vec{w} + \vec{w}^\perp$.
\end{theorem}

\begin{proof}
This is an example of an ``existence and uniqueness'' theorem, so there are two things to prove.  If we have an orthogonal basis $\{\vec{f}_{1}, \vec{f}_{2}, \dots, \vec{f}_{m}\}$ for $W$, then it is easy to show that our orthogonal decomposition exists for $\vec{x}$. We let $\vec{w}=\mbox{proj}_W(\vec{x})$, which is clearly in $W$, and we let  $\vec{w}^\perp = \vec{x} - \vec{w}$, and we have $\vec{w} + \vec{w}^\perp = \vec{w} + (\vec{x} - \vec{w}) = \vec{x}$, so we need to see that $\vec{w}^\perp \in W^\perp$.

By Theorem~\ref{th:023783}~\ref{th:023783c}, it suffices to show that $\vec{w}^\perp$ is orthogonal to each of the basis vectors $\vec{f}_i, i=1,\ldots,m$.  We compute for $i=1,\ldots,m$
\begin{align*}
    \vec{f}_i \dotp \vec{w}^\perp 
    &= \vec{f}_i \dotp (\vec{x} - \vec{w}) \\
    &= \vec{f}_i \dotp \vec{x} -  \vec{f}_i \dotp  \left(\frac{\vec{x} \dotp \vec{f}_{1}}{\norm{\vec{f}_{1}}^2}\vec{f}_{1} + \frac{\vec{x} \dotp \vec{f}_{2}}{\norm{\vec{f}_{2}}^2}\vec{f}_{2}+ \dots +\frac{\vec{x} \dotp \vec{f}_{m}}{\norm{\vec{f}_{m}}^2}\vec{f}_{m}\right) \\
    &= \vec{f}_i \dotp \vec{x} - \left(\frac{\vec{x} \dotp \vec{f}_{1}}{\norm{\vec{f}_{1}}^2}\vec{f}_i \dotp\vec{f}_{i} \right) = \vec{f}_i \dotp \vec{x} - (\vec{x} \dotp \vec{f}_i) = 0.
\end{align*}
This proves that $\vec{w}^\perp \in W^\perp$.

The reason we need to prove this decomposition is unique is because we started with the orthogonal basis $\{\vec{f}_{1}, \vec{f}_{2}, \dots, \vec{f}_{m}\}$ for $W$, but what would happen if we chose a different orthogonal basis?  

Suppose that $\{\vec{f}_1^\prime, \vec{f}_2^\prime, \dots, \vec{f}_m^\prime \}$  is another orthogonal basis of $W$, and let
\begin{equation*}
\vec{w}^{\prime} = \left(\frac{\vec{x} \dotp \vec{f}^{\prime}_{1}}{\norm{\vec{f}^{\prime}_{1}}^2}\right)\vec{f}^{\prime}_{1} + \left(\frac{\vec{x} \dotp \vec{f}^{\prime}_{2}}{\norm{\vec{f}^{\prime}_{2}}^2}\right)\vec{f}^{\prime}_{2} + \dots +\left(\frac{\vec{x} \dotp \vec{f}^{\prime}_{m}}{\norm{\vec{f}^{\prime}_{m}}^2}\right)\vec{f}^{\prime}_{m}
\end{equation*}
As before, $\vec{w}^{\prime} \in W$ and $\vec{x} - \vec{w}^{\prime} \in W^\perp$, and we must show that $\vec{w}^{\prime} = \vec{w}$. To see this, write the vector $\vec{w} - \vec{w}^\prime$ as follows:
\begin{equation*}
\vec{w} - \vec{w}^{\prime} = (\vec{x} - \vec{w}^{\prime}) - (\vec{x} - \vec{w})
\end{equation*}
This vector is in $W$ (because $\vec{w}$ and $\vec{w}^\prime$ are in $W$) and it is in $W^\perp$ (because $\vec{x} - \vec{w}^\prime$ and $\vec{x} - \vec{w}$ are in $W^\perp$), and so it must be the zero vector (it is orthogonal to itself!). This means $\vec{w}^\prime = \vec{w}$ as desired.
\end{proof}

\begin{example}\label{ex:OrthogDecomp}
Let $W$ be a subspace given by $W = \mbox{span}\left(\begin{bmatrix}
  1 \\ 0 \\ 1 \\ 0
  \end{bmatrix},
  \begin{bmatrix}
  0 \\ 1 \\ 0 \\ 2
\end{bmatrix}\right)$, and let $\vec{x}=\begin{bmatrix}
  1 \\ 2 \\ 3 \\ 4
  \end{bmatrix}$.  Write $\vec{x}$ as the sum of a vector in $W$ and a vector in $W^\perp$.
  
  \begin{explanation}
   Following the notation of Theorem \ref{th:OrthoDecomp}, we will write $\vec{x} = \vec{w} + \vec{w}^\perp$, where $\vec{w}=\mbox{proj}_W(\vec{x})$ and $\vec{w}^\perp = \vec{x} - \vec{w}$.  Let $\vec{f}_1=\begin{bmatrix}
  1 \\ 0 \\ 1 \\ 0
  \end{bmatrix}$ and let $\vec{f}_2=\begin{bmatrix}
  0 \\ 1 \\ 0 \\ 2
  \end{bmatrix}$.  We observe that we have the good fortune that $\vec{f}_1,\vec{f}_2$ is an orthogonal basis for $W$ (otherwise, our first step would be to use the Gram-Schmidt procedure to create an orthogonal basis for $W$).  We compute:
$$\vec{w}=\mbox{proj}_W(\vec{x})
      =\vec{x}-\mbox{proj}_{\vec{f}_1}(\vec{x})-\mbox{proj}_{\vec{f}_2}(\vec{x})
      = \begin{bmatrix}
  1 \\ 2 \\ 3 \\ 4
  \end{bmatrix} - \frac{4}{2}\begin{bmatrix}
  1 \\ 0 \\ 1 \\ 0
  \end{bmatrix} - \frac{10}{5}\begin{bmatrix}
  0 \\ 1 \\ 0 \\ 2
  \end{bmatrix} = \begin{bmatrix}
  2 \\ 2 \\ 2 \\ 4
  \end{bmatrix},$$
  and then $\vec{w}^\perp=\begin{bmatrix}
  1 \\ 2 \\ 3 \\ 4
  \end{bmatrix} - \begin{bmatrix}
  2 \\ 2 \\ 2 \\ 4
  \end{bmatrix} = \begin{bmatrix}
  -1 \\ 0 \\ 1 \\ 0
  \end{bmatrix}.$

  This gives us
  $$\vec{x}=\vec{w}+\vec{w}^\perp=\begin{bmatrix}2\\2\\2\\4\end{bmatrix}+\begin{bmatrix}-1\\0\\1\\0\end{bmatrix}$$
  \end{explanation}
  \end{example}
  
The final theorem of this section shows that projection onto a subspace of $\RR^n$ is actually a linear transformation from $\RR^n$ to $\RR^n$.

\begin{theorem}\label{th:ProjLinTran}
Let $W$ be a fixed subspace of $\RR^n$. If we define $T : \RR^n \to \RR^n$ by
\begin{equation*}
T(\vec{x}) = \mbox{proj}_W(\vec{x}) \quad \mbox{ for all }\vec{x}\mbox{ in }\RR^n
\end{equation*}
\begin{enumerate}
\item\label{th:ProjLinTran_a} $T$ is a linear transformation.  (See \href{https://ximera.osu.edu/oerlinalg/LinearAlgebra/LTR-0010/main}{Introduction to Linear Transformations}.)

\item\label{th:ProjLinTran_b} $\mbox{im}(T)$ is $W$ and $\mbox{ker}(T)$ is $ W^\perp$.  (See \href{https://ximera.osu.edu/oerlinalg/LinearAlgebra/LTR-0050/main}{Image and Kernel of a Linear Transformation}.)

\item\label{th:ProjLinTran_c} $\mbox{dim}(W) + \mbox{dim}(W^\perp) = n$.

\end{enumerate}
\end{theorem}

\begin{proof}
If $W = \{\vec{0}\}$, then $W^\perp = \RR^n$, and so $T(\vec{x}) = \vec{0}$ for all $\vec{x}$. Thus $T = 0$ is the zero (linear) operator, so \ref{th:ProjLinTran_a}, \ref{th:ProjLinTran_b}, and \ref{th:ProjLinTran_c} hold. Hence assume that $W \neq \{\vec{0}\}$.

\begin{enumerate}
\item If $\{\vec{q}_{1}, \vec{q}_{2}, \dots, \vec{q}_{m}\}$ is an orthonormal basis of $W$, then
\begin{equation}\label{orthonormalUeq}
T(\vec{x}) = (\vec{x} \dotp \vec{q}_{1})\vec{q}_{1} + (\vec{x} \dotp \vec{q}_{2})\vec{q}_{2} + \dots + (\vec{x} \dotp \vec{q}_{m})\vec{q}_{m} \quad \mbox{ for all }\vec{x}\in \RR^n
\end{equation}
by the definition of the projection. Thus $T$ is a linear transformation because
\begin{equation*}
(\vec{x} + \vec{y}) \dotp \vec{q}_{i} = \vec{x} \dotp \vec{q}_{i} + \vec{y} \dotp \vec{q}_{i} \quad \mbox{ and } \quad (r\vec{x}) \dotp \vec{q}_{i} = r(\vec{x} \dotp \vec{q}_{i}) \quad \mbox{ for each } i.
\end{equation*}

\item %This proof is correct, but perhaps written at a higher level than most of the proofs in our book.
We have $\mbox{im}(T)$ is a subset of $W$ by (\ref{orthonormalUeq}) because each $\vec{q}_{i}$ is in $W$. But if $\vec{x}$ is in $W$, then $\vec{x} = T(\vec{x})$ by (\ref{orthonormalUeq}) and Theorem \ref{th:fourierexpansion} applied to the space $W$. This shows that $W$ is a subset of $\mbox{im}(T)$, so $\mbox{im}(T)$ is $W$.

Now suppose that $\vec{x}$ is in $W^\perp$. Then $\vec{x} \dotp \vec{q}_{i} = 0$ for each $i$ (again because each $\vec{q}_{i}$ is in $W$) so $\vec{x}$ is in $\mbox{ker}(T)$ by (\ref{th:023783}). Hence $W^\perp$ is in $\mbox{ker}(T)$. On the other hand, Theorem~\ref{th:023783} shows that $\vec{x} - T(\vec{x})$ is in $W^\perp$ for all $\vec{x}$ in $\RR^n$, and it follows that $\mbox{ker}(T)$ is in $W^\perp$. Hence $\mbox{ker}(T)$ is $W^\perp$, proving \ref{th:ProjLinTran_b}.

\item This follows from \ref{th:ProjLinTran_a}, \ref{th:ProjLinTran_b}, and the Rank-Nullity theorem (Theorem~\ref{th:ranknullityforT}).
\end{enumerate}
\end{proof}

\section*{Practice Problems}
\begin{problem}\label{prob:Uperp}
Solve the linear system in Example \ref{ex:023829} and use your result to find a basis for $W^\perp$ if $W = \mbox{span}\left(\begin{bmatrix}1\\ -1\\ 2\\ 0\end{bmatrix}, \begin{bmatrix}1\\ 0\\ -2\\ 3\end{bmatrix}\right)$ in $\RR^4$.
\end{problem}

\begin{problem}\label{prob:brokenInteractive}
In this problem we return to the GeoGebra interactive in Exploration \ref{exp:discoverortho}, and we consider the case where the matrix $A$ has rank 1 (which Exploration \ref{exp:discoverortho} could not handle).  This time, the sliders define row 1 of matrix $A$, and row 2 will be 2 times row 1.  Follow the prompts in the interactive to visualize $\mbox{row}(A)$ and $\mbox{null}(A)$.  What relationships do you observe between $\mbox{row}(A)$ and $\mbox{null}(A)$?

\pdfOnly{
Access GeoGebra interactives through the online version of this text at 

\href{https://ximera.osu.edu/oerlinalg}{https://ximera.osu.edu/oerlinalg}.
}

\begin{onlineOnly}
    \begin{center}
\geogebra{tyntjmdp}{950}{800}
\end{center}
\end{onlineOnly}
\end{problem}

\begin{problem}\label{prob:finishex4subspaces}
In this problem you are asked to finish Example \ref{ex:4subspaces}.  More specifically, for the matrix $$A=\begin{bmatrix}2&-1&1&-4&1\\1&0&3&3&0\\-2&1&-1&5&2\\4&-1&7&2&1\end{bmatrix},$$ show that $\mbox{null}(A^T) = (\mbox{col}(A))^\perp$.  It may be helpful to consult \href{https://ximera.osu.edu/oerlinalg/LinearAlgebra/VSP-0040/main}{Subspaces of $\RR^n$ Associated with Matrices}, where we found a basis for the column space of this matrix.
\end{problem}

\emph{Problems \ref{OrthoDecomp2}-\ref{OrthoDecomp6}}

In each case, write $\vec{x}$ as $\vec{x} = \vec{w} + \vec{w}^\perp$, where $\vec{w}=\mbox{proj}_W(\vec{x})$ and $\vec{w}^\perp = \vec{x} - \vec{w}$.
%In each case, write $\vec{x}$ as the sum of a vector in $W$ and a vector in $W^\perp$.

\begin{problem}\label{OrthoDecomp2}
$\vec{x} = \begin{bmatrix}2\\ 1\\ 6\end{bmatrix}$, $W = \mbox{span}\left(\begin{bmatrix}3\\ -1\\ 2\end{bmatrix}, \begin{bmatrix}2\\ 0\\ -3\end{bmatrix}\right)$
\end{problem}

\begin{problem}\label{OrthoDecomp4}
$\vec{x} = \begin{bmatrix}2\\ 0\\ 1\\ 6\end{bmatrix}$, $W = \mbox{span}\left(\begin{bmatrix}1\\ 1\\ 1\\ 1\end{bmatrix}, \begin{bmatrix}1\\ 1\\ -1\\ -1\end{bmatrix}, \begin{bmatrix}1\\ -1\\ 1\\ -1\end{bmatrix}\right)$
\end{problem}

\begin{problem}\label{OrthoDecomp6}
$\vec{x} = \begin{bmatrix}a\\ b\\ c\\ d\end{bmatrix}$, $W = \mbox{span}\left(\begin{bmatrix}1\\ -1\\ 2\\ 0\end{bmatrix}, \begin{bmatrix}-1\\ 1\\ 1\\ 1\end{bmatrix}\right)$
\end{problem}

	

\begin{problem}\label{Uperp}
Let $W = \mbox{span}\left(\vec{w}_{1}, \vec{w}_{2}, \dots, \vec{w}_{k}\right)$, $\vec{w}_{i}\in \RR^n$, and let $A$ be the $k \times n$ matrix with the $\vec{w}_{i}$ as rows.


\begin{enumerate}
\item Show that $W^\perp = \{\vec{x} \mid  \vec{x}\in \RR^n, A\vec{x}^{T} = \vec{0}\}$.

\item Use part (a) to find $W^\perp$ if $$W = \mbox{span}\left(\begin{bmatrix}1\\ -1\\ 2\\ 1\end{bmatrix}, \begin{bmatrix}1\\ 0\\ -1\\ 1\end{bmatrix}\right).$$
%ANSWER:  $W^\perp = \mbox{span}\left((1, 3, 1, 0), (-1, 0, 0, 1)\right)$
\end{enumerate}

\end{problem}

\begin{problem}\label{prob:8_1_6}
\begin{enumerate}
\item Prove part \ref{th:023783a} of Theorem~\ref{th:023783}.

\item Prove part \ref{th:023783b} of Theorem~\ref{th:023783}.

\end{enumerate}
\end{problem}

\begin{problem} \label{prob:8_1_7}
Let $W$ be a subspace of $\RR^n$. If $\vec{x}$ in $\RR^n$ can be written in any way at all as $\vec{x} = \vec{p} + \vec{q}$ with $\vec{p}$ in $W$ and $\vec{q}$ in $W^\perp$, show that necessarily $\vec{p} = \mbox{proj}_W(\vec{x})$.
\end{problem}

\begin{problem}\label{prob:8_1_8}
Let $W$ be a subspace of $\RR^n$ and let $\vec{x}$ be a vector in $\RR^n$. Using Practice Problem \ref{prob:8_1_7}, or otherwise, show that $\vec{x}$ is in $W$ if and only if $\vec{x} = \mbox{proj}_W(\vec{x})$.

\begin{hint}
Write $\vec{w} = \mbox{proj}_W(\vec{x})$. Then $\vec{w}$ is in $W$ by definition. If $\vec{x}$ is $W$, then $\vec{x} - \vec{w}$ is in $W$. But $\vec{x} - \vec{w}$ is also in $W^\perp$, so $\vec{x} - \vec{w}$ is in $W \cap U^\perp = \{\vec{0}\}$. Thus $\vec{x} = \vec{w}$.
\end{hint}
\end{problem}


\begin{problem}\label{prob:8_1_10}
If $W$ is a subspace of $\RR^n$, show that $\mbox{proj}_W(\vec{x}) = \vec{x}$ for all $\vec{x}$ in $W$.

\begin{hint}
Let $\{\vec{q}_{1}, \vec{q}_{2}, \dots , \vec{q}_{m}\}$ be an orthonormal basis of $W$. If $\vec{x}$ is in $W$ the expansion theorem gives $\vec{x} = (\vec{x} \dotp \vec{q}_{1})\vec{q}_{1} + (\vec{x} \dotp \vec{q}_{2})\vec{q}_{2} + \dots  + (\vec{x} \dotp \vec{q}_{m})\vec{q}_{m} = \mbox{proj}_W(\vec{x})$.
\end{hint}
\end{problem}

\begin{problem}\label{prob:8_1_11}
If $W$ is a subspace of $\RR^n$, show that $\vec{x} = \mbox{proj}_W(\vec{x}) + \mbox{proj}_{W^\perp}(\vec{x})$ for all $\vec{x}$ in $\RR^n$.
\end{problem}

\begin{problem}\label{prob:8_1_12}
If $\{\vec{v}_{1}, \dots, \vec{v}_{n}\}$ is an orthogonal basis of $\RR^n$ and $W = \mbox{span}\left(\vec{v}_{1}, \dots, \vec{v}_{m}\right)$, $m<n$, show that $W^\perp = \mbox{span}\left(\vec{v}_{m + 1}, \dots, \vec{v}_{n}\right)$.
\end{problem}


\section*{Text Source} This section was adapted from the second part of Section 8.1 of Keith Nicholson's \href{https://open.umn.edu/opentextbooks/textbooks/linear-algebra-with-applications}{\it Linear Algebra with Applications}. (CC-BY-NC-SA)

W. Keith Nicholson, {\it Linear Algebra with Applications}, Lyryx 2018, Open Edition, p. 415--423. 

%\section*{Example Source}
Example \ref{ex:OrthogDecomp}  was adapted from Example 4.148  of Ken Kuttler's \href{https://open.umn.edu/opentextbooks/textbooks/a-first-course-in-linear-algebra-2017}{\it A First Course in Linear Algebra}. (CC-BY)

Ken Kuttler, {\it  A First Course in Linear Algebra}, Lyryx 2017, Open Edition, p. 249. 

%W. Keith Nicholson, {\it Linear Algebra with Applications}, Lyryx 2018, Open Edition, p. 346, 350

%\section*{Exercise Source}
%Practice Problems \ref{prob:linindabstractvsp1}, \ref{prob:linindabstractvsp2} and \ref{prob:linindabstractvsp3} are Exercises 6.3(a)(b)(c) from Keith Nicholson's \href{https://open.umn.edu/opentextbooks/textbooks/linear-algebra-with-applications}{\it Linear Algebra with Applications}. (CC-BY-NC-SA)

%W. Keith Nicholson, {\it Linear Algebra with Applications}, Lyryx 2018, Open Edition, p. 351



\end{document}