\documentclass{ximera}
%% You can put user macros here
%% However, you cannot make new environments

\listfiles

\graphicspath{
{./}
{./LTR-0070/}
{./VEC-0060/}
{./APP-0020/}
}

\usepackage{tikz}
\usepackage{tkz-euclide}
\usepackage{tikz-3dplot}
\usepackage{tikz-cd}
\usetikzlibrary{shapes.geometric}
\usetikzlibrary{arrows}
%\usetkzobj{all}
\pgfplotsset{compat=1.13} % prevents compile error.

%\renewcommand{\vec}[1]{\mathbf{#1}}
\renewcommand{\vec}{\mathbf}
\newcommand{\RR}{\mathbb{R}}
\newcommand{\dfn}{\textit}
\newcommand{\dotp}{\cdot}
\newcommand{\id}{\text{id}}
\newcommand\norm[1]{\left\lVert#1\right\rVert}
 
\newtheorem{general}{Generalization}
\newtheorem{initprob}{Exploration Problem}

\tikzstyle geometryDiagrams=[ultra thick,color=blue!50!black]

%\DefineVerbatimEnvironment{octave}{Verbatim}{numbers=left,frame=lines,label=Octave,labelposition=topline}



\usepackage{mathtools}


 \title{Additional Exercises for Ch 2} \license{CC BY-NC-SA 4.0}

\begin{document}

\begin{abstract}
\end{abstract}
\maketitle

\section*{Additional Exercises for Ch 2: Systems of Linear Equations}

\subsection*{Review Exercises}

\begin{problem}\label{prb:2.2}
Graphically, find the point of intersection of the two lines $
3x+y=3$ and $x+2y=1.$ That is, graph each line
and see where they intersect.

 $$x=\answer{1},y=\answer{0}$$
\end{problem}

\begin{problem}\label{prb:2.3} You have a system of $k$ equations in two variables, $k\geq 2$.
Explain the geometric significance of

\begin{enumerate}
\item No solution.
\item A unique solution.
\item An infinite number of solutions.
\end{enumerate}
\end{problem}

\begin{problem}\label{prb:2.4}
Consider the following augmented matrix in which $\ast $ denotes an
arbitrary number and $\blacksquare $ denotes a nonzero number. Determine
whether the given augmented matrix is consistent. If consistent, is the
solution unique?
\begin{equation*}
\left[
\begin{array}{ccccc|c}
\blacksquare & \ast & \ast & \ast & \ast & \ast \\
0 & \blacksquare & \ast & \ast & 0 & \ast \\
0 & 0 & \blacksquare & \ast & \ast & \ast \\
0 & 0 & 0 & 0 & \blacksquare & \ast
\end{array}
\right] 
\end{equation*}

The solution \wordChoice{\choice[correct]{exists} \choice{does not exist}} and is \wordChoice{\choice{unique} \choice[correct]{not unique}}.
\end{problem}

\begin{problem}\label{prb:2.5}
Consider the following augmented matrix in which $\ast $ denotes an arbitrary
number and $\blacksquare $ denotes a nonzero number. Determine whether the
given augmented matrix is consistent. If consistent, is the solution unique?
\begin{equation*}
\left[
\begin{array}{ccc|c}
\blacksquare & \ast & \ast & \ast \\
0 & \blacksquare & \ast & \ast \\
0 & 0 & \blacksquare & \ast
\end{array}
\right]
\end{equation*}
The solution \wordChoice{\choice[correct]{exists} \choice{does not exist}} and is \wordChoice{[correct]\choice{unique} \choice{not unique}}.
\end{problem}


\begin{problem}\label{prb:2.6}
Consider the following augmented matrix in which $\ast $ denotes an arbitrary
number and $\blacksquare $ denotes a nonzero number. Determine whether the
given augmented matrix is consistent. If consistent, is the solution unique?
\begin{equation*}
\left[
\begin{array}{ccccc|c}
\blacksquare & \ast & \ast & \ast & \ast & \ast \\
0 & \blacksquare & 0 & \ast & 0 & \ast \\
0 & 0 & 0 & \blacksquare & \ast & \ast \\
0 & 0 & 0 & 0 & \blacksquare & \ast
\end{array}
\right]
\end{equation*}
The solution \wordChoice{\choice[correct]{exists} \choice{does not exist}} and is \wordChoice{\choice{unique} \choice[correct]{not unique}}.
\end{problem}

\begin{problem}\label{prb:2.7}
Consider the following augmented matrix in which $\ast $ denotes an arbitrary
number and $\blacksquare $ denotes a nonzero number. Determine whether the
given augmented matrix is consistent. If consistent, is the solution unique?
\begin{equation*}
\left[
\begin{array}{ccccc|c}
\blacksquare & \ast & \ast & \ast & \ast & \ast \\
0 & \blacksquare & \ast & \ast & 0 & \ast \\
0 & 0 & 0 & 0 & \blacksquare & 0 \\
0 & 0 & 0 & 0 & \ast & \blacksquare
\end{array}
\right]
\end{equation*}
Click the arrow to see the answer. 
\begin{expandable}
The third equation implies that $x_5 = 0$.  The fourth equation implies that $x_5 \ne 0$.  We conclude that the system is inconsistent.
\end{expandable}
\end{problem}

\begin{problem}\label{prb:2.8}
Suppose a system of equations has fewer equations than variables. Will such a system necessarily be consistent? If so, explain why and if not, give an example which is not consistent.

Click the arrow to see the answer.
\begin{expandable}
No. Consider $x+y+z=2$ and $x+y+z=1.$
\end{expandable}
\end{problem}

\begin{problem}\label{prb:2.9}
If a system of equations has more equations than variables, can it
have a solution? If so, give an example and if not, tell why not.

Click the arrow to see the answer. 
\begin{expandable}
These can
have a solution. For example, $x+y=1,2x+2y=2,3x+3y=3$ even has an infinite
set of solutions.
\end{expandable}
\end{problem}

\begin{problem}\label{prb:2.10}
Find $h$ such that
\begin{equation*}
\left[
\begin{array}{rr|r}
2 & h & 4 \\
3 & 6 & 7
\end{array}
\right]
\end{equation*}
is the augmented matrix of an \textit{inconsistent} system.

$h=\answer{4}$

\end{problem}

\begin{problem}\label{prb:2.11}
Find $h$ such that
\begin{equation*}
\left[
\begin{array}{rr|r}
1 & h & 3 \\
2 & 4 & 6
\end{array}
\right]
\end{equation*}
is the augmented matrix of a \textit{consistent} system.

Click the arrow to see the answer. 
\begin{expandable}
 Any $h$ will work.
\end{expandable}
\end{problem}

\begin{problem}\label{prb:2.12}
Find $h$ such that
\begin{equation*}
\left[
\begin{array}{rr|r}
1 & 1 & 4 \\
3 & h & 12
\end{array}
\right]
\end{equation*}
is the augmented matrix of a \textit{consistent} system.

Click the arrow to see the answer. 
\begin{expandable}
 Any $h$ will work.
\end{expandable}
\end{problem}


\begin{problem}\label{prb:2.13}
Choose $h$ and $k$ such that the augmented matrix shown has each of the following:
\begin{enumerate}
\item one solution
\item no solution
\item infinitely many solutions
\end{enumerate}
\begin{equation*}
\left[
\begin{array}{rr|r}
1 & h & 2 \\
2 & 4 & k
\end{array}
\right]
\end{equation*}
Click the arrow to see the answer. 
\begin{expandable}
If $h\neq 2$ there will be a unique solution for any $k$. If $h=2$ and $%
k\neq 4,$ there are no solutions. If $h=2$ and $k=4,$ then there are
infinitely many solutions.
\end{expandable}
\end{problem}


\begin{problem}\label{prb:2.14}
Choose $h$ and $k$ such that the augmented matrix shown has each of the following:
\begin{enumerate}
\item one solution
\item no solution
\item infinitely many solutions
\end{enumerate}
\begin{equation*}
\left[
\begin{array}{rr|r}
1 & 2 & 2 \\
2 & h & k
\end{array}
\right]
\end{equation*}
Click the arrow to see the answer. 
\begin{expandable}
If $h\neq 4,$ then there is exactly one solution. If $h=4$ and $k\neq 4,$
then there are no solutions. If $h=4$ and $k=4,$ then there are infinitely
many solutions.
\end{expandable}
\end{problem}


\begin{problem}\label{prb:2.15}
Determine if the system is consistent. If so, is the solution unique?
$$\begin{array}{ccccccccc}
      x & +&2y&+&z&-&w&= &2 \\
	 x& -&y&+&z&+&w&=&1\\
     2x& +&y&-&z&&&=&1\\
     4x&+&2y&+&z&&&=&5
    \end{array}$$

Click the arrow to see the answer. 
\begin{expandable}
There is no solution. The system is inconsistent. You can see this from the
augmented matrix. $\mbox{rref}\left(\left[
\begin{array}{rrrr|r}
1 & 2 & 1 & -1 & 2 \\
1 & -1 & 1 & 1 & 1 \\
2 & 1 & -1 & 0 & 1 \\
4 & 2 & 1 & 0 & 5
\end{array}
\right]\right) = \left[
\begin{array}{rrrr|r}
1 & 0 & 0 &  \frac{1}{3} & 0 \\
0 & 1 & 0 & - \frac{2}{3} & 0 \\
0 & 0 & 1 & 0 & 0 \\
0 & 0 & 0 & 0 & 1
\end{array}
\right] .$
\end{expandable}
\end{problem}

\begin{problem}\label{prb:2.16}
Determine if the system is consistent. If so, is the solution unique?

$$\begin{array}{ccccccccc}
      x & +&2y&+&z&-&w&= &2 \\
	 x& -&y&+&z&+&w&=&0\\
     2x& +&y&-&z&&&=&1\\
     4x&+&2y&+&z&&&=&3
    \end{array}$$

Click the arrow to see the answer. 
\begin{expandable}
Solution is: $ w=\frac{3}{2}y-1, x=\frac{2}{3}-\frac{1}{2}y, z=\frac{1}{3} $
\end{expandable}
\end{problem}

\begin{problem}\label{prb:2.17} Determine which matrices are in reduced row-echelon form.

\begin{enumerate}
\item $\left[
\begin{array}{rrr}
1 & 2 & 0 \\
0 & 1 & 7
\end{array}
\right] $ \wordChoice{\choice{yes} \choice[correct]{no}} 

\item $\left[
\begin{array}{rrrr}
1 & 0 & 0 & 0 \\
0 & 0 & 1 & 2 \\
0 & 0 & 0 & 0
\end{array}
\right] $ \wordChoice{\choice[correct]{yes} \choice{no}}

\item $\left[
\begin{array}{rrrrrr}
1 & 1 & 0 & 0 & 0 & 5 \\
0 & 0 & 1 & 2 & 0 & 4 \\
0 & 0 & 0 & 0 & 1 & 3
\end{array}
\right] $ \wordChoice{\choice[correct]{yes} \choice{no}}

\end{enumerate}
\end{problem}

\begin{problem}\label{prb:2.18} Row reduce the following matrix to row echelon form. Then continue to obtain the reduced row echelon form.

\begin{equation*}
\left[
\begin{array}{rrrr}
2 & -1 & 3 & -1 \\
1 & 0 & 2 & 1 \\
1 & -1 & 1 & -2
\end{array}
\right]
\end{equation*}

Click the arrow to see the answer. 
\begin{expandable}
$\begin{bmatrix}1 & 0&2&1\\0&1&1&3\\0&0&0&0\end{bmatrix}$
\end{expandable}

\end{problem}

\begin{problem}\label{prb:2.19} Row reduce the following matrix to row echelon form. Then continue to obtain the reduced row echelon form.
\begin{equation*}
\left[
\begin{array}{rrrr}
0 & 0 & -1 & -1 \\
1 & 1 & 1 & 0 \\
1 & 1 & 0 & -1
\end{array}
\right]
\end{equation*}

Click the arrow to see the answer. 
\begin{expandable}
$\begin{bmatrix}1 & 1&0&-1\\0&0&1&1\\0&0&0&0\end{bmatrix}$
\end{expandable}
\end{problem}

\begin{problem}\label{prb:2.20} Row reduce the following matrix to row echelon form. Then continue to obtain the reduced row echelon form.
\begin{equation*}
\left[
\begin{array}{rrrr}
3 & -6 & -7 & -8 \\
1 & -2 & -2 & -2 \\
1 & -2 & -3 & -4
\end{array}
\right]
\end{equation*}

Click the arrow to see the answer. 
\begin{expandable}
$\begin{bmatrix}1 & -2&0&2\\0&0&1&2\\0&0&0&0\end{bmatrix}$
\end{expandable}

\end{problem}

\begin{problem}\label{prb:2.21} Row reduce the following matrix to row echelon form. Then continue to obtain the reduced row echelon form.
\begin{equation*}
\left[
\begin{array}{rrrr}
2 & 4 & 5 & 15 \\
1 & 2 & 3 & 9 \\
1 & 2 & 2 & 6
\end{array}
\right]
\end{equation*}

Click the arrow to see the answer. 
\begin{expandable}
$\begin{bmatrix}1 & 2&0&0\\0&0&1&3\\0&0&0&0\end{bmatrix}$
\end{expandable}
\end{problem}

\begin{problem}\label{prb:2.22} Row reduce the following matrix to row echelon form. Then continue to obtain the reduced row echelon form.
\begin{equation*}
\left[
\begin{array}{rrrr}
4 & -1 & 7 & 10 \\
1 & 0 & 3 & 3 \\
1 & -1 & -2 & 1
\end{array}
\right]
\end{equation*}

Click the arrow to see the answer. 
\begin{expandable}
$\begin{bmatrix}1 & 0&3&3\\0&1&5&2\\0&0&0&0\end{bmatrix}$
\end{expandable}
\end{problem}

\begin{problem}\label{prb:2.23} Row reduce the following matrix to row echelon form. Then continue to obtain the reduced row echelon form.
\begin{equation*}
\left[
\begin{array}{rrrr}
3 & 5 & -4 & 2 \\
1 & 2 & -1 & 1 \\
1 & 1 & -2 & 0
\end{array}
\right]
\end{equation*}

Click the arrow to see the answer. 
\begin{expandable}
$\begin{bmatrix}1 & 0&-3&-1\\0&1&1&1\\0&0&0&0\end{bmatrix}$
\end{expandable}
\end{problem}

\begin{problem}\label{prb:2.24} Row reduce the following matrix to row echelon form. Then continue to obtain the reduced row echelon form.
\begin{equation*}
\left[
\begin{array}{rrrr}
-2 & 3 & -8 & 7 \\
1 & -2 & 5 & -5 \\
1 & -3 & 7 & -8
\end{array}
\right]
\end{equation*}

Click the arrow to see the answer. 
\begin{expandable}
$\begin{bmatrix}1 & 0&1&1\\0&1&-2&3\\0&0&0&0\end{bmatrix}$
\end{expandable}
\end{problem}

\begin{problem}\label{prb:2.25} Find the solution of the system whose augmented matrix is
\begin{equation*}
\left[
\begin{array}{rrr|r}
1 & 2 & 0 & 2 \\
1 & 3 & 4 & 2 \\
1 & 0 & 2 & 1
\end{array}
\right]
\end{equation*}

Click the arrow to see the answer. 
\begin{expandable}
$x=\frac{6}{5}, y=\frac{2}{5}, z=-\frac{1}{10}$
\end{expandable}
\end{problem}

\begin{problem}\label{prb:2.26} Find the solution of the system whose augmented matrix is
\begin{equation*}
\left[
\begin{array}{rrr|r}
1 & 2 & 0 & 2 \\
2 & 0 & 1 & 1 \\
3 & 2 & 1 & 3
\end{array}
\right]
\end{equation*}
Click the arrow to see the answer. 
\begin{expandable}
The reduced row echelon form is $\left[
\begin{array}{rrr|r}
1 & 0 &  \frac{1}{2} &  \frac{1}{2} \\
0 & 1 & - \frac{1}{4} &  \frac{3}{4} \\
0 & 0 & 0 & 0
\end{array}
\right] .$ Therefore, the solution is of the form $z=t,y=\frac{3}{4}+t\left(
\frac{1}{4}\right) ,x=\frac{1}{2}-\frac{1}{2}t$ where $t\in \mathbb{R}$.
\end{expandable}
\end{problem}

\begin{problem}\label{prb:2.27} Find the solution of the system whose augmented matrix is
\begin{equation*}
\left[
\begin{array}{rrr|r}
1 & 1 & 0 & 1 \\
1 & 0 & 4 & 2
\end{array}
\right]
\end{equation*}
Click the arrow to see the answer. 
\begin{expandable}
The reduced row echelon form is $\left[
\begin{array}{rrr|r}
1 & 0 & 4 & 2 \\
0 & 1 & -4 & -1
\end{array}
\right] $ and so the solution is $z=t,y=4t,x=2-4t.$
\end{expandable}
\end{problem}

\begin{problem}\label{prb:2.28} Find the solution of the system whose augmented matrix is
\begin{equation*}
\left[
\begin{array}{rrrrr|r}
1 & 0 & 2 & 1 & 1 & 2 \\
0 & 1 & 0 & 1 & 2 & 1 \\
1 & 2 & 0 & 0 & 1 & 3 \\
1 & 0 & 1 & 0 & 2 & 2
\end{array}
\right]
\end{equation*}
Click the arrow to see the answer. 
\begin{expandable}
The reduced row echelon form is $\left[
\begin{array}{rrrrr|r}
1 & 0 & 0 & 0 & 9 & 3 \\
0 & 1 & 0 & 0 & -4 & 0 \\
0 & 0 & 1 & 0 & -7 & -1 \\
0 & 0 & 0 & 1 & 6 & 1
\end{array}
\right] $ and so $x_{5}=t,x_{4}=1-6t,x_{3}=-1+7t,x_{2}=4t,x_{1}=3-9t$.
\end{expandable}
\end{problem}

\begin{problem}\label{prb:2.29} Find the solution of the system whose augmented matrix is
\begin{equation*}
\left[
\begin{array}{rrrrr|r}
1 & 0 & 2 & 1 & 1 & 2 \\
0 & 1 & 0 & 1 & 2 & 1 \\
0 & 2 & 0 & 0 & 1 & 3 \\
1 & -1 & 2 & 2 & 2 & 0
\end{array}
\right]
\end{equation*}
Click the arrow to see the answer. 
\begin{expandable}
The reduced row echelon form is $\left[
\begin{array}{rrrrr|r}
1 & 0 & 2 & 0 & - \frac{1}{2} &  \frac{5}{2} \\
0 & 1 & 0 & 0 &  \frac{1}{2} &  \frac{3}{2} \\
0 & 0 & 0 & 1 &  \frac{3}{2} & - \frac{1}{2} \\
0 & 0 & 0 & 0 & 0 & 0
\end{array}
\right] $. Therefore, let $x_{5}=t,x_{3}=s.$ Then the other variables are
given by $x_{4}=-\frac{1}{2}-\frac{3}{2}t,x_{2}=\frac{3}{2}-t\frac{1}{2}
,,x_{1}=\frac{5}{2}+\frac{1}{2}t-2s.$
\end{expandable}
\end{problem}

\begin{problem}\label{prb:2.30} Find the solution to the system of equations, $7x+14y+15z=22,
$ $2x+4y+3z=5,$ and $3x+6y+10z=13.$
 \begin{align*}
 x&=\answer{1}-\answer{2}t\\
 y&=t\\
 z&=\answer{1}\\
 \end{align*}
\end{problem}

\begin{problem}\label{prb:2.31} Find the solution to the system of equations, $3x-y+4z=6,$
$y+8z=0,$ and $-2x+y=-4.$
 \begin{align*}
 x&=\answer{2}-\answer{4}t\\
 y&=\answer{-8}t\\
 z&=t\\
 \end{align*}
\end{problem}

\begin{problem}\label{prb:2.32} Find the solution to the system of equations, $9x-2y+4z=-17,
$ $13x-3y+6z=-25,$ and $-2x-z=3.$
 \begin{align*}
 x&=\answer{-1}\\
 y&=\answer{2}\\
 z&=\answer{-1}\\
 \end{align*}
\end{problem}

\begin{problem}\label{prb:2.33} Find the solution to the system of equations,
$65x+84y+16z=546,$ $81x+105y+20z=682,$ and $84x+110y+21z=713.$
 \begin{align*}
 x&=\answer{2}\\
 y&=\answer{4}\\
 z&=\answer{5}\\
 \end{align*}
\end{problem}

\begin{problem}\label{prb:2.34} Find the solution to the system of equations,
$8x+2y+3z=-3,8x+3y+3z=-1,$ and $4x+y+3z=-9.$
 \begin{align*}
 x&=\answer{1}\\
 y&=\answer{2}\\
 z&=\answer{-5}\\
 \end{align*}
\end{problem}

\begin{problem}\label{prb:2.35} Find the solution to the system of equations,
$-8x+2y+5z=18,-8x+3y+5z=13,$ and $-4x+y+5z=19.$
 \begin{align*}
 x&=\answer{-1}\\
 y&=\answer{-5}\\
 z&=\answer{4}\\
 \end{align*}
\end{problem}

\begin{problem}\label{prb:2.36} Find the solution to the system of equations, $3x-y-2z=3,$
$y-4z=0,$ and $-2x+y=-2.$
 \begin{align*}
 x&=\answer{1}-\answer{2}t\\
 y&=\answer{4}t\\
 z&=t\\
 \end{align*}
\end{problem}

\begin{problem}\label{prb:2.37} Find the solution to the system of equations,
$-9x+15y=66,-11x+18y=79$, $-x+y=4$, and $z=3$.
 \begin{align*}
 x&=\answer{1}\\
 y&=\answer{5}\\
 z&=\answer{3}\\
 \end{align*}
\end{problem}

\begin{problem}\label{prb:2.38} Find the solution to the system of equations, $-19x+8y=-108,$
$-71x+30y=-404,$ $-2x+y=-12,$ $4x+z=14.$
 \begin{align*}
 x&=\answer{4}\\
 y&=\answer{-4}\\
 z&=\answer{-2}\\
 \end{align*}
\end{problem}

\begin{problem}\label{prb:2.39} Suppose a system of equations has fewer equations than variables and
you have found a solution to this system of equations. Is it possible that
your solution is the only one? Explain.

\begin{hint}
No. Consider what happens when you solve the system $x+y+z=2$ and $x-y-z=1$.
\end{hint}
\end{problem}

\begin{problem}\label{prb:2.40} Suppose a system of linear equations has a $2\times 4$ augmented
matrix and the last column is a pivot column. Could the system of linear
equations be consistent? Explain.

Click the arrow to see the answer. 
\begin{expandable}
 No. If the last column is a pivot column, then the last row looks like this: $[0 0 0 | 1]$.  This would lead to $0=1.$
\end{expandable}
\end{problem}

\begin{problem}\label{prb:2.41} Suppose the coefficient matrix of a system of $n$ equations with $n$
variables has the property that every column is a pivot column. Does it
follow that the system of equations must have a solution? If so, must the
solution be unique? Explain.

The solution \wordChoice{\choice[correct]{exists} \choice{does not exist}} and is \wordChoice{\choice[correct]{unique} \choice{not unique}}.
\end{problem}

\begin{problem}\label{prb:2.42} Suppose there is a unique solution to a system of linear equations.
What must be true of the pivot columns in the augmented matrix?

Click the arrow to see the answer. 
\begin{expandable}
The last column must not be a pivot column. The remaining columns must each be pivot
columns.
\end{expandable}
\end{problem}


\begin{problem}\label{prb:2.43} The steady state temperature, $u$, of a plate solves Laplace's
equation, $\Delta u=0.$ One way to approximate the solution is to divide the plate into a square mesh and require the temperature
at each node to equal the average of the temperature at the four adjacent
nodes. In the following picture, the numbers represent the observed
temperature at the indicated nodes. Find the temperature at
the interior nodes, indicated by $x,y,z,$ and $w$. One of the equations is
$z=\frac{1}{4}\left( 10+0+w+x\right) $.

\begin{center}
   \begin{tikzpicture}[scale=1]
\node[red] at (-0.3, 1)   (a) {$20$};
\node[red] at (-0.3, 2)   (a) {$20$};
    \node[red] at (3.3, 1)   (b) {$0$};
     \node[red] at (3.3, 2)   (b) {$0$};
     \node[red] at (1, -0.3)   (c) {$10$};
      \node[red] at (2, -0.3)   (c) {$10$};
      \node[red] at (1, 3.3)   (c) {$30$};
      \node[red] at (2, 3.3)   (c) {$30$};
      \node[] at (1.2, 1.2)   (c) {$x$};
      \node[] at (2.2, 1.2)   (c) {$z$};
      \node[] at (1.2, 2.2)   (c) {$y$};
      \node[] at (2.2, 2.2)   (c) {$w$};
  \draw[-] (0,2)--(3,2);
  \draw[-] (0,1)--(3,1);
  \draw[-] (1,0)--(1,3);
  \draw[-] (2,0)--(2,3);
  \fill[] (1,1) circle (0.05cm); 
  \fill[] (1,2) circle (0.05cm); 
   \fill[] (2,1) circle (0.05cm); 
  \fill[] (2,2) circle (0.05cm); 
   \fill[] (0,1) circle (0.05cm); 
  \fill[] (0,2) circle (0.05cm); 
   \fill[] (3,1) circle (0.05cm); 
  \fill[] (3,2) circle (0.05cm); 
   \fill[] (1,0) circle (0.05cm); 
  \fill[] (1,3) circle (0.05cm); 
   \fill[] (2,0) circle (0.05cm); 
  \fill[] (2,3) circle (0.05cm); 
    \end{tikzpicture}
\end{center}

\begin{hint}
You need 
\begin{align*}
&\frac{1}{4}\left( 20+30+w+x\right) -y=0 \\
&\frac{1}{4}\left( y+30+0+z\right) -w=0 \\
&\frac{1}{4}\left( 20+y+z+10\right) -x=0 \\
&\frac{1}{4}\left( x+w+0+10\right) -z=0
\end{align*}
\end{hint}
 \begin{align*}
 w&=\answer{15}\\
 x&=\answer{15}\\
 y&=\answer{20}\\
 z&=\answer{10}\\
 \end{align*}
\end{problem}

\begin{problem}\label{prb:2.44} Find the rank of the following matrix.
\begin{equation*}
\left[
\begin{array}{rrrr}
4 & -16 & -1 & -5 \\
1 & -4 & 0 & -1 \\
1 & -4 & -1 & -2
\end{array}
\right]
\end{equation*}
Answer: $\answer{2}$
\end{problem}

\begin{problem}\label{prb:2.45a} Find the rank of the following matrix.
\begin{equation*}
\left[
\begin{array}{rrrr}
1 & 5 & 3 & 7 \\
4 & 9 & 6 & 2 \\
8 & 0 & 5 & 3 \\
2 & 7 & 4 & 1
\end{array}
\right]
\end{equation*}
Answer: $\answer{4}$
\end{problem}

\begin{problem}\label{prb:2.45} Find the rank of the following matrix.
\begin{equation*}
\left[
\begin{array}{rrrr}
3 & 6 & 5 & 12 \\
1 & 2 & 2 & 5 \\
1 & 2 & 1 & 2
\end{array}
\right]
\end{equation*}
Answer: $\answer{2}$
\end{problem}

\begin{problem}\label{prb:2.46} Find the rank of the following matrix.
\begin{equation*}
\left[
\begin{array}{rrrrr}
0 & 0 & -1 & 0 & 3 \\
1 & 4 & 1 & 0 & -8 \\
1 & 4 & 0 & 1 & 2 \\
-1 & -4 & 0 & -1 & -2
\end{array}
\right]
\end{equation*}
Answer: $\answer{3}$
\end{problem}

\begin{problem}\label{prb:2.47} Find the rank of the following matrix.
\begin{equation*}
\left[
\begin{array}{rrrr}
4 & -4 & 3 & -9 \\
1 & -1 & 1 & -2 \\
1 & -1 & 0 & -3
\end{array}
\right]
\end{equation*}
Answer: $\answer{2}$
\end{problem}

\begin{problem}\label{prb:2.48} Find the rank of the following matrix.
\begin{equation*}
\left[
\begin{array}{rrrrr}
2 & 0 & 1 & 0 & 1 \\
1 & 0 & 1 & 0 & 0 \\
1 & 0 & 0 & 1 & 7 \\
1 & 0 & 0 & 1 & 7
\end{array}
\right]
\end{equation*}
Answer: $\answer{3}$
\end{problem}

\begin{problem}\label{prb:2.49} Find the rank of the following matrix.
\begin{equation*}
\left[
\begin{array}{rrr}
4 & 15 & 29 \\
1 & 4 & 8 \\
1 & 3 & 5 \\
3 & 9 & 15
\end{array}
\right]
\end{equation*}
Answer: $\answer{2}$
\end{problem}

\begin{problem}\label{prb:2.50} Find the rank of the following matrix.
\begin{equation*}
\left[
\begin{array}{rrrrr}
0 & 0 & -1 & 0 & 1 \\
1 & 2 & 3 & -2 & -18 \\
1 & 2 & 2 & -1 & -11 \\
-1 & -2 & -2 & 1 & 11
\end{array}
\right]
\end{equation*}
Answer: $\answer{3}$
\end{problem}

\begin{problem}\label{prb:2.51} Find the rank of the following matrix.
\begin{equation*}
\left[
\begin{array}{rrrrr}
1 & -2 & 0 & 3 & 11 \\
1 & -2 & 0 & 4 & 15 \\
1 & -2 & 0 & 3 & 11 \\
0 & 0 & 0 & 0 & 0
\end{array}
\right]
\end{equation*}
Answer: $\answer{2}$
\end{problem}

\begin{problem}\label{prb:2.52} Find the rank of the following matrix.
\begin{equation*}
\left[
\begin{array}{rrr}
-2 & -3 & -2 \\
1 & 1 & 1 \\
1 & 0 & 1 \\
-3 & 0 & -3
\end{array}
\right]
\end{equation*}
Answer: $\answer{2}$
\end{problem}

\begin{problem}\label{prb:2.53} Find the rank of the following matrix.
\begin{equation*}
\left[
\begin{array}{rrrrr}
4 & 4 & 20 & -1 & 17 \\
1 & 1 & 5 & 0 & 5 \\
1 & 1 & 5 & -1 & 2 \\
3 & 3 & 15 & -3 & 6
\end{array}
\right]
\end{equation*}
Answer: $\answer{2}$
\end{problem}

\begin{problem}\label{prb:2.54} Find the rank of the following matrix.
\begin{equation*}
\left[
\begin{array}{rrrrr}
-1 & 3 & 4 & -3 & 8 \\
1 & -3 & -4 & 2 & -5 \\
1 & -3 & -4 & 1 & -2 \\
-2 & 6 & 8 & -2 & 4
\end{array}
 \right]
\end{equation*}
Answer: $\answer{2}$
\end{problem}

\begin{problem}\label{prb:2.55} Suppose $A$ is an $m\times n$ matrix. Explain why the rank of $A$ is
always no larger than $\min \left( m,n\right).$

Click the arrow to see the answer. 
\begin{expandable}
It is because you cannot
have more leading 1's than columns and you cannot have more leading 1's than rows.
%more than $\min \left( m,n\right) $ nonzero rows in the reduced row echelon form. Recall that the number of pivot columns is the same as the
%number of nonzero rows from the description of this reduced row echelon form.
\end{expandable}
\end{problem}

\begin{problem}\label{prb:2.56} State whether each of the following scenarios is possible for the
system of equations $\left[A |\vec{b}\right]$. If possible, describe the solution set.
That is, tell whether there exists a unique solution, no solution or
infinitely many solutions. Here, $\left[ A |\vec{b} \right]$ denotes the augmented matrix.

\begin{enumerate}
\item $A$ is a $5\times 6$ matrix, $\mbox{rank}\left( A\right) =4$ and
$\mbox{rank}\left[ A |\vec{b} \right] =4.$

Click the arrow to see the answer. 
\begin{expandable}
Infinite solution set.
\end{expandable}

\item $A$ is a $3\times 4$ matrix, $\mbox{rank}\left( A\right) =3$ and
$\mbox{rank}\left[ A |\vec{b} \right] =2.$

Click the arrow to see the answer. 
\begin{expandable}
This surely can't happen. If you add in another column, the rank does not get smaller.
\end{expandable}

\item $A$ is a $4\times 2$ matrix, $\mbox{rank}\left( A\right) =4$ and
$\mbox{rank}\left[ A |\vec{b} \right] =4.$

Click the arrow to see the answer. 
\begin{expandable}
You can't have the rank equal 4 if you only have two columns.
\end{expandable}

\item $A$ is a $5\times 5$ matrix, $\mbox{rank}\left( A\right) =4$ and
$\mbox{rank}\left[ A |\vec{b} \right] =5.$

Click the arrow to see the answer. 
\begin{expandable}
This is possible, but $\mbox{rref} \left[A | \vec{b}\right]$ contains a row $[0 0 0 0 0 | 1]$, making the system inconsistent.
\end{expandable}

\item $A$ is a $4\times 2$ matrix, $\mbox{rank}\left( A\right) =2$ and
$\mbox{rank}\left[ A |\vec{b} \right] =2$.

Click the arrow to see the answer. 
\begin{expandable}
In this case, there is a
unique solution.
\end{expandable}
\end{enumerate}
\end{problem}

\subsection*{Challenge Exercises}

\begin{problem}\label{prb:possible}
We show in \href{https://ximera.osu.edu/oerlinalg/LinearAlgebra/RRN-0030/main}{Planes in $\RR^3$} that the graph of an equation $ax + by + cz = d$ is a plane in space when not all of $a$, $b$, and $c$ are zero.

\begin{enumerate}
\item By examining the possible positions of planes in space, show that three equations in three variables can have zero, one, or infinitely many solutions.
 %ANSWER  No. If the corresponding planes are parallel and distinct, there is no
%solution. Otherwise they either coincide or have a whole common line of
%solutions, that is, at least one parameter.

\item Can two equations in three variables have a unique solution? Give reasons for your answer.
\end{enumerate}
\end{problem}

\begin{problem}\label{prb:systems_review}
Find all solutions to the following systems of linear equations.

\begin{enumerate}
\item

$\begin{array}{rlrlrlrcr}
	  x_1 & + &  x_2 & + &  x_3 & - &  x_4 & = &  3 \\
	 3x_1 & + & 5x_2 & - & 2x_3 & + &  x_4 & = &  1 \\
	-3x_1 & - & 7x_2 & + & 7x_3 & - & 5x_4 & = &  7 \\
	  x_1 & + & 3x_2 & - & 4x_3 & + & 3x_4 & = & -5 \\
\end{array}$

Click the arrow to see the answer.
\begin{expandable}
$x_1 = \frac{1}{10}(-6s - 6t + 16)$, $x_2 = \frac{1}{10}(4s - t + 1)$, $x_3 = s$, $x_4 = t$
\end{expandable}
\item

$\begin{array}{rlrlrlrcr}
	  x_1 & + &  4x_2 & - &   x_3 & + &   x_4 & = &  2 \\
	 3x_1 & + &  2x_2 & + &   x_3 & + &  2x_4 & = &  5 \\
	  x_1 & - &  6x_2 & + &  3x_3 &   &       & = &  1 \\
	  x_1 & + & 14x_2 & - &  5x_3 & + &  2x_4 & = &  3 \\
\end{array}$

\end{enumerate}

\end{problem}

\begin{problem}\label{prb:no._sol}
In each case find (if possible) conditions on $a$, $b$, and $c$ such that the system has zero, one, or infinitely many solutions.
\begin{enumerate}
\item

$\begin{array}{rlrlrcr}
	  x & + &  2y & - &  4z & = &  4 \\
	 3x & - &   y & + & 13z & = &  2 \\
	 4x & + &   y & + & a^2z & = & a + 3 \\
\end{array}$

Click the arrow to see the answer.
\begin{expandable}
If $a = 1$, no solution. If $a = 2$, $x = 2 - 2t$, $y = -t$, $z = t$. If $a \neq 1$ and $a \neq 2$, the unique  solution is $x = \frac{8 - 5a}{3(a - 1)}$, $y = \frac{-2 - a}{3(a - 1)}$, $z = \frac{a + 2}{3}$
\end{expandable}
\item

$\begin{array}{rlrlrcr}
	  x & + &   y & + &  3z & = &  a \\
	 ax & + &   y & + &  5z & = &  4 \\
	  x & + &  ay & + &  4z & = &  a
\end{array}$
\end{enumerate}
\end{problem}

\begin{problem}\label{prb:elementaryrowops}
Show that any two rows of a matrix can be interchanged by using the other two elementary row operations.  (This shows that one of the elementary row operations is ``redundant".)

Click the arrow to see the answer.
\begin{expandable}
$\begin{bmatrix}
	R_1 \\
	R_2
\end{bmatrix}
\to \begin{bmatrix}
	R_1 + R_2 \\
	R_2
\end{bmatrix}
\to
\begin{bmatrix}
	R_1 + R_2 \\
	-R_1
\end{bmatrix}
\to
\begin{bmatrix}
	R_2 \\
	-R_1
\end{bmatrix}
\to
\begin{bmatrix}
	R_2 \\
	R_1
\end{bmatrix}$
\end{expandable}
\end{problem}

\begin{problem}\label{prb:rref_nonsingular}
If $ad \neq bc$, show that $\begin{bmatrix}
a & b \\
c & d
\end{bmatrix}$ has reduced row-echelon form $\begin{bmatrix}
 1 & 0 \\
 0 & 1
 \end{bmatrix}$.
\end{problem}

\begin{problem}\label{prb:coeffs}
Find $a$, $b$, and $c$ so that the system
\begin{equation*}
\begin{array}{rlrlrcr}
	 x & + &  ay & + &  cz & = &  0 \\
	bx & + &  cy & - &  3z & = &  1 \\
	ax & + &  2y & + &  bz & = &  5
\end{array}
\end{equation*}
has the solution $x = 3$, $y = -1$, $z = 2$.

$a = \answer{1}$, $b = \answer{2}$, $c = \answer{-1}$

\end{problem}

\begin{problem}\label{prb:complex}
Solve the system
$$
\begin{array}{rlrlrcr}
	 x & + &  2y & + &  2z & = & -3 \\
	2x & + &   y & + &   z & = & -4 \\
	 x & - &   y & + &  iz & = &  i
\end{array}
$$
where $i^{2} = -1$. 
\end{problem}

\begin{problem}\label{prb:real_complex}
Show that the \textit{real} system
\begin{equation*}
\begin{array}{rlrlrcr}
	 x & + &   y & + &   z & = &  5 \\
	2x & - &   y & - &   z & = &  1 \\
	-3x& + &  2y & + &  2z & = &  0
\end{array}
\end{equation*}
has a \textit{complex} solution: $x = 2$, $y = i$, $z = 3 - i$ where $i^{2} = -1$. Explain. What happens when such a real system has a unique solution?

Click the arrow to see the answer.
\begin{expandable}
The (real) solution is $x = 2$, $y = 3 - t$, $z = t$ where $t$ is a parameter. The given complex solution occurs when $t = 3 - i$ is complex. If the real system has a unique solution, that solution is real because the coefficients and constants are all real.
\end{expandable}
\end{problem}

\begin{problem}\label{prb:pills}
A man is ordered by his doctor to take $5$ units of vitamin A, $13$ units of vitamin B, and $23$ units of vitamin C each day. Three brands of vitamin pills are available, and the number of units of each vitamin per pill are shown in the accompanying table.

$$
\begin{array}{cccc}
\hline
		\textbf{Brand} &  \textbf{A} & \textbf{B} & \textbf{C} \\ \hline
		\textbf{1} & 1 & 2 & 4 \\
		\textbf{2} & 1 & 1 & 3 \\
		\textbf{3} & 0 & 1 & 1 \\ \hline
 \end{array}
$$

\begin{enumerate}
\item Find all combinations of pills that provide exactly the required amount of vitamins (no partial pills allowed).

\item If brands 1, 2, and 3 cost 3 cents, 2 cents, and 5 cents \ per pill, respectively, find the least expensive treatment.

$\answer{5}$ of brand 1, $\answer{0}$ of brand 2, $\answer{3}$ of brand 3

\end{enumerate}
\end{problem}

\begin{problem}\label{prb:tables}
A restaurant owner plans to use $x$ tables seating $4$, $y$ tables seating $6$, and $z$ tables seating $8$, for a total of $20$ tables. When fully occupied, the tables seat $108$ customers. If only half of the $x$ tables, half of the $y$ tables, and one-fourth of the $z$ tables are used, each fully occupied, then $46$ customers will be seated. Find $x$, $y$, and $z$.

\end{problem}

\begin{problem}\label{prb:2x2rref_forms}

\begin{enumerate}
\item Show that a matrix with two rows and two columns that is in reduced row-echelon form must have one of the following forms:
\begin{equation*}
\begin{bmatrix}
1 & 0 \\
0 & 1
\end{bmatrix}
\begin{bmatrix}
0 & 1 \\
0 & 0
\end{bmatrix}
\begin{bmatrix}
0 & 0 \\
0 & 0
\end{bmatrix}
\begin{bmatrix}
1 & * \\
0 & 0
\end{bmatrix}
\end{equation*}

Click the arrow to see the answer. 
\begin{expandable}
 The leading $1$ in the first row must be in column 1 or 2 or not exist.
\end{expandable}

\item List the seven reduced row-echelon forms for matrices with two rows and three columns.

\item List the four reduced row-echelon forms for matrices with three rows and two columns.

\end{enumerate}
\end{problem}

\begin{problem}\label{prb:amusement}
An amusement park charges \$$7$ for adults, \$$2$ for youths, and \$$0.50$ for
children. If $150$ people enter and pay a total of \$$100$, find the numbers
of adults, youths, and children.

\begin{hint}
These numbers are nonnegative \textit{integers}.
\end{hint}

\end{problem}

\begin{problem}\label{prb:quadratic}
Solve the following system of equations for $x$ and $y$.

\begin{equation*}
\begin{array}{rlrlrcr}
	  x^2 & + &   xy & - &   y^2 & = &  1 \\
	 2x^2 & - &   xy & + &  3y^2 & = & 13 \\
	  x^2 & + &  3xy & + &  2y^2 & = &  0 \\
\end{array}
\end{equation*}

\begin{hint}
These equations are linear in the new variables $x_{1} = x^{2}$, $x_{2} = xy$, and $x_{3} = y^{2}$.
\end{hint}
\end{problem}


\subsection*{Octave Exercises}
\begin{problem}\label{oct:rref}
Use Octave to check your work on Problems \ref{prb:2.18} to \ref{prb:2.38}.  The command $\mbox{rref}(A)$ is an easy way to obtain the reduced row echelon form of a matrix.  It is more challenging to go through the steps of Gaussian elimination, but it can be done.

To use Octave, go to the \href{https://sagecell.sagemath.org/}{Sage Math Cell Webpage}, copy the code below into the cell, select OCTAVE as the language, and press EVALUATE.  The code does the first two steps of \ref{prb:2.18}.  See if you can add one more command which will put the matrix into row echelon form.

\begin{verbatim}
% Solving systems using Gaussian elimination
A=[2 -1 3 -1; 1 0 2 1; 1 -1 1 -2];
R=rref(A)
A
A(2,:)=-1/2*A(1,:)+A(2,:)
A(3,:)=-1/2*A(1,:)+A(3,:)
\end{verbatim}
\end{problem}

\begin{problem}\label{oct:rank}
Use Octave to check your work on Problems \ref{prb:2.44} to \ref{prb:2.54}.  The command $\mbox{rank}(A)$ is an easy way to obtain the rank of a matrix.  Of course, you can also use $\mbox{rref}(A)$ and you obtain even more information!  Problem \ref{prb:2.44} is entered in the code below.  

To use Octave, go to the \href{https://sagecell.sagemath.org/}{Sage Math Cell Webpage}, copy the code below into the cell, select OCTAVE as the language, and press EVALUATE.

\begin{verbatim}
% Computing the rank of a matrix
A=[4 -16 -1 -5; 1 -4 0 -1; 1 -4 -1 -2]
r=rank(A)
R=rref(A)
\end{verbatim}
\end{problem}

\section*{Bibliography}
The Review Exercises come from the end of Chapter 1 of Ken Kuttler's \href{https://open.umn.edu/opentextbooks/textbooks/a-first-course-in-linear-algebra-2017}{\it A First Course in Linear Algebra}. (CC-BY)

Ken Kuttler, {\it  A First Course in Linear Algebra}, Lyryx 2017, Open Edition, pp. 42--49. 

The Challenge Exercises come from the end of Chapter 1 of Keith Nicholson's \href{https://open.umn.edu/opentextbooks/textbooks/linear-algebra-with-applications}{\it Linear Algebra with Applications}. (CC-BY-NC-SA)

W. Keith Nicholson, {\it Linear Algebra with Applications}, Lyryx 2018, Open Edition, pp. 33--34. 

\end{document}