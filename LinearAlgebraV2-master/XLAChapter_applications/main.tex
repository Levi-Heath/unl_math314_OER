\documentclass{ximera}
%% You can put user macros here
%% However, you cannot make new environments

\listfiles

\graphicspath{
{./}
{./LTR-0070/}
{./VEC-0060/}
{./APP-0020/}
}

\usepackage{tikz}
\usepackage{tkz-euclide}
\usepackage{tikz-3dplot}
\usepackage{tikz-cd}
\usetikzlibrary{shapes.geometric}
\usetikzlibrary{arrows}
%\usetkzobj{all}
\pgfplotsset{compat=1.13} % prevents compile error.

%\renewcommand{\vec}[1]{\mathbf{#1}}
\renewcommand{\vec}{\mathbf}
\newcommand{\RR}{\mathbb{R}}
\newcommand{\dfn}{\textit}
\newcommand{\dotp}{\cdot}
\newcommand{\id}{\text{id}}
\newcommand\norm[1]{\left\lVert#1\right\rVert}
 
\newtheorem{general}{Generalization}
\newtheorem{initprob}{Exploration Problem}

\tikzstyle geometryDiagrams=[ultra thick,color=blue!50!black]

%\DefineVerbatimEnvironment{octave}{Verbatim}{numbers=left,frame=lines,label=Octave,labelposition=topline}



\usepackage{mathtools}


\title{Applications} \license{CC BY-NC-SA 4.0}

\begin{document}
\begin{abstract}
Links to 70+ applications.
\end{abstract}
\maketitle
 
One of the beautiful aspects of linear algebra is its wide variety of applications, and it is our belief that a first course in linear algebra should include exposure to many of them.  In this chapter we have six different applications, but there are so many others that may catch the interest of a particular student.  Here we list a few resources where you can find other applications.

\begin{enumerate}
    \item Feryal Alayont and Steven Schlicker, {\it  Linear Algebra and Applications: An Inquiry-Based Approach}, Scholarworks at GVSU, 2019, Open Edition, available at \href{https://scholarworks.gvsu.edu/books/21/}{https://scholarworks.gvsu.edu/books/21/}

    In this free book, each chapter opens with an application that is tackled later in the chapter.  This makes for 37 different applications of elementary linear algebra!

    \item \href{https://www.analyticsvidhya.com/blog/2019/07/10-applications-linear-algebra-data-science/}{10 Powerful Applications of Linear Algebra in Data Science (with Multiple Resources)}

    This website gives plenty of ideas and additional resources that can be explored while studying linear algebra.

    \item
    Jim Hefferon, {\it  Linear Algebra }, freely available at \href{https://hefferon.net/linearalgebra/}{https://hefferon.net/linearalgebra/}

    In this free book, each chapter contains 4-5 applications after covering the other content.  This makes for 23 different applications of elementary linear algebra!

\end{enumerate}





 
\end{document}