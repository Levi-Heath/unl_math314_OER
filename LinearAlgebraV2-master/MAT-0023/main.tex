\documentclass{ximera}
%% You can put user macros here
%% However, you cannot make new environments

\listfiles

\graphicspath{
{./}
{./LTR-0070/}
{./VEC-0060/}
{./APP-0020/}
}

\usepackage{tikz}
\usepackage{tkz-euclide}
\usepackage{tikz-3dplot}
\usepackage{tikz-cd}
\usetikzlibrary{shapes.geometric}
\usetikzlibrary{arrows}
%\usetkzobj{all}
\pgfplotsset{compat=1.13} % prevents compile error.

%\renewcommand{\vec}[1]{\mathbf{#1}}
\renewcommand{\vec}{\mathbf}
\newcommand{\RR}{\mathbb{R}}
\newcommand{\dfn}{\textit}
\newcommand{\dotp}{\cdot}
\newcommand{\id}{\text{id}}
\newcommand\norm[1]{\left\lVert#1\right\rVert}
 
\newtheorem{general}{Generalization}
\newtheorem{initprob}{Exploration Problem}

\tikzstyle geometryDiagrams=[ultra thick,color=blue!50!black]

%\DefineVerbatimEnvironment{octave}{Verbatim}{numbers=left,frame=lines,label=Octave,labelposition=topline}



\usepackage{mathtools}


 \title{Block Matrix Multiplication} \license{CC BY-NC-SA 4.0}

\begin{document}
\begin{abstract}

\end{abstract}
\maketitle

\section*{Block Matrix Multiplication}

It is often useful to partition a matrix into smaller matrices, called \dfn{blocks}. A matrix viewed in this way is said to be \dfn{partitioned into blocks}.  For example, each column of a matrix can be considered to be a block.  Writing a matrix $B$ in the form
\begin{equation*}
B = \begin{bmatrix}
\vec{b}_{1} & \vec{b}_{2} & \ldots & \vec{b}_{k}
\end{bmatrix} \mbox{ where the } \vec{b}_{j} \mbox{ are the columns of } B
\end{equation*}
is a block partition of $B$. 

Here is another example.  Consider matrix $A$.
\begin{equation*}
A = \left[ \begin{array}{rrrrr}
1 & 0 & 0 & 0 & 0 \\
0 & 1 & 0 & 0 & 0 \\
2 & -1 & 4 & 2 & 1 \\
3 & 1 & -1 & 7 & 5
\end{array} \right]
\end{equation*}
There is a natural way to partition $A$. Observe the $2\times 2$ identity matrix, $I_2$, in the upper left corner.  There is also a $2\times 3$ zero matrix, $O_{23}$, in the upper right corner.  We will take advantage of these features to partition $A$ as follows.

\begin{equation*}
A = \left[ \begin{array}{rr|rrr}
1 & 0 & 0 & 0 & 0 \\
0 & 1 & 0 & 0 & 0 \\
\hline
2 & -1 & 4 & 2 & 1 \\
3 & 1 & -1 & 7 & 5
\end{array} \right] = \left[ \begin{array}{cc}
I_{2} & O_{23} \\
P & Q
\end{array} \right]
\end{equation*}

This notation is particularly useful when we are multiplying two matrices because the product can be computed in block form.

Consider matrix $B$.
\begin{equation*}
B = \left[ \begin{array}{rr}
4 & -2 \\
5 & 6 \\
7 & 3 \\
-1 & 0 \\
1 & 6
\end{array} \right]
\end{equation*}
We will compute the product $AB$ by ordinary matrix multiplication, using blocks as entries. The only requirement is that the blocks be compatible. That is, the sizes of the blocks must be such that all matrix products of blocks that occur make sense. This means that the number of columns in each block of $A$ must equal the number of rows in the corresponding block of $B$.

To find the product $AB$, we need to partition $B$ so that block $I_2$ corresponds to a $2\times 2$ block of $B$, and block $O_{23}$ of $A$ corresponds to a $3\times 2$ block of $B$.  We partition $B$ as follows.
\begin{equation*}
B = \left[ \begin{array}{rr}
4 & -2 \\
5 & 6 \\
\hline
7 & 3 \\
-1 & 0 \\
1 & 6
\end{array} \right] = \left[ \begin{array}{c}
X \\
Y
\end{array} \right]
\end{equation*}

Then,

\begin{equation*}
AB = \left[ \begin{array}{cc}
I & O \\
P & Q
\end{array} \right] \left[ \begin{array}{c}
X \\
Y
\end{array} \right] = \left[ \begin{array}{c}
IX + OY \\
PX + QY
\end{array} \right] = \left[ \begin{array}{c}
X \\
PX + QY
\end{array} \right] = \left[ \begin{array}{rr}
4 & -2 \\
5 & 6 \\
\hline
30 & 8 \\
8 & 27
\end{array} \right]
\end{equation*}
This is easily checked to be the product $AB$, computed in the conventional manner.




\begin{theorem}\label{th:blockmatmult}
If matrices $A$ and $B$ are partitioned compatibly into blocks, the product $AB$ can be computed by matrix multiplication using blocks as entries.
\end{theorem}
We omit the proof.

%We have been using two cases of block multiplication. If $B = \begin{bmatrix}
%\vec{b}_{1} & \vec{b}_{2} & \ldots & \vec{b}_{k}
%\end{bmatrix}$ is a matrix where the $\vec{b}_{j}$ are the columns of $B$, and if the matrix product $AB$ is defined, then we have
%\begin{equation*}
%AB = A \begin{bmatrix}
%\vec{b}_{1} & \vec{b}_{2} & \cdots & \vec{b}_{k}
%\end{bmatrix} = \begin{bmatrix}
%A\vec{b}_{1} & A\vec{b}_{2} & \cdots & A\vec{b}_{k}
%\end{bmatrix}
%\end{equation*}
%As another illustration,
%\begin{equation*}
%B\vec{x} = \begin{bmatrix}
%\vec{b}_{1} & \vec{b}_{2} & \cdots & \vec{b}_{k}
%\end{bmatrix} \begin{bmatrix}
%x_{1} \\
%x_{2} \\
%\vdots \\
%x_{k}
%\end{bmatrix} = x_{1}\vec{b}_{1} + x_{2}\vec{b}_{2} + \cdots + x_{k}\vec{b}_{k}
%\end{equation*}
%where $\vec{x}$ is any $k \times 1$ column matrix.





%\begin{theorem}{}{003738}
%Suppose matrices $A = \left[ \begin{array}{cc}
%B & X \\
%0 & C
%\end{array} \right]$
% and $A_{1} = \left[ \begin{array}{cc}
% B_{1} & X_{1} \\
% 0 & C_{1}
% \end{array} \right]$ are partitioned as shown where $B$ and $B_{1}$ are square matrices of the same size, and $C$ and $C_{1}$ are also square of the same size. These are compatible partitionings and block multiplication gives
%\begin{equation*}
%AA_{1} = \left[ \begin{array}{cc}
%B & X \\
%0 & C
%\end{array} \right] \left[ \begin{array}{cc}
%B_{1} & X_{1} \\
%0 & C_{1}
%\end{array} \right] = \left[ \begin{array}{cc}
%BB_{1} & BX_{1} + XC_{1} \\
%0 & CC_{1}
%\end{array} \right]
%\end{equation*}
%\end{theorem}

%\begin{example}{}{003746}
%Obtain a formula for $A^{k}$ where $A = \left[ \begin{array}{cc}
%I & X \\
%0 & 0
%\end{array} \right]$
% is square and $I$ is an identity matrix.


%\begin{solution}
%  We have $A^{2} = \left[ \begin{array}{cc}
%  I & X \\
%  0 & 0
%  \end{array} \right] \left[ \begin{array}{cc}
%  I & X \\
%  0 & 0
%  \end{array} \right] = \left[ \begin{array}{cc}
%  I^{2} & IX + X0 \\
%  0 & 0^{2}
%  \end{array} \right] = \left[ \begin{array}{cc}
%  I & X \\
%  0 & 0
%  \end{array} \right] = A$. Hence $A^{3} = AA^{2} = AA = A^{2} = A$. Continuing in this way, we see that $A^{k} = A$ for every $k \geq 1$.
%\end{solution}
%\end{example}

Block multiplication has theoretical uses, as we shall see later. It is also useful in computing products of matrices when using a computer with limited memory capacity. The matrices are partitioned into blocks in such a way that each product of blocks can be handled. Then the blocks are stored in auxiliary memory and their products are computed one by one.

\section*{Practice Problems}

\begin{problem}\label{prob:blockmatmult1}
Compute $AB$, using the indicated block partitioning.
\begin{equation*}
A = \left[ \begin{array}{rr|rr}
2 & -1 & 3 & 1 \\
1 & 0  & 1 & 2 \\
\hline
0 & 0 & 1 & 0 \\
0 & 0 & 0 & 1
\end{array} \right] \quad
B = \left[ \begin{array}{rr|r}
1 & 2 & 0 \\
-1 & 0  & 0 \\
\hline
0 & 5 & 1 \\
1 & -1 & 0 
\end{array} \right]
\end{equation*}

Answer:
\begin{equation*}
AB = \left[ \begin{array}{rr|r}
\answer{4} & \answer{18} & \answer{3} \\
\answer{3} & \answer{5}  & \answer{1} \\
\hline
\answer{0} & \answer{5} & \answer{1} \\
\answer{1} & \answer{-1} & \answer{0}
\end{array} \right] 
\end{equation*}

\end{problem}

\begin{problem}\label{prob:block_powers}
In each case give formulas for all powers $A, A^{2}, A^{3}, \dots$ of $A$ using the block decomposition indicated.


\begin{enumerate}
\item
$A = \left[ \begin{array}{r|rr}
1 & 0 & 0 \\
\hline
1 & 1  & -1\\
1 & -1 & 1
\end{array} \right]
$

Click the arrow to see the answer.
\begin{expandable}
$
A^{2k} = \left[ \begin{array}{rc|rr}
1 & -2k & 0 & 0 \\
0 & 1 & 0 & 0 \\
\hline
0 & 0 & 1 & 0 \\
0 & 0 & 0 & 1
\end{array} \right]$ for $k = 0, 1, 2, \dots$
\end{expandable}

\item
$A = \left[ \begin{array}{rr|rr}
1 & -1 & 2 & -1 \\
0 & 1 & 0 & 0 \\
\hline
0 & 0 & -1 & 1 \\
0 & 0 & 0 & 1
\end{array} \right]
$

Click the arrow to see the answer.
\begin{expandable}
$A^{2k + 1} = A^{2k}A = \left[ \begin{array}{rc|rr}
1 & -(2k + 1) & 2 & -1 \\
0 & 1 & 0 & 0 \\
\hline
0 & 0 & -1 & 1 \\
0 & 0 & 0 & 1
\end{array} \right]$ for $k = 0, 1, 2, \dots$
\end{expandable}
\end{enumerate}
\end{problem}


\begin{problem}\label{prob:blockmatmult_vars}
Compute the following using block multiplication (all blocks are $k \times k$).

\begin{enumerate}
\item
$\left[ \begin{array}{rr}
I & X \\
-Y & I
\end{array} \right] \left[ \begin{array}{rr}
I & 0 \\
Y & I
\end{array} \right]
$

\item %challenging
$\left[ \begin{array}{cc}
I & X \\
0 & -I
\end{array} \right]^{n}
$ any $n \geq 1$

\item %challenging
$\left[ \begin{array}{cc}
0 & X \\
I & 0
\end{array} \right]^{n}
$ any $n \geq 1$
\end{enumerate}

Click the arrow to see the answer.
\begin{expandable}
$\left[ \begin{array}{cc}
X^{m} & 0 \\
0 & X^{m}
\end{array} \right]$ if $n = 2m$;
$\left[ \begin{array}{cc}
0 & X^{m + 1} \\
X^{m} & 0
\end{array} \right]$ if $n = 2m + 1$
\end{expandable}


\end{problem}

\section*{Text Source}
The text in this section is an adaptation of Section 2.3 of Keith Nicholson's \href{https://open.umn.edu/opentextbooks/textbooks/linear-algebra-with-applications}{\it Linear Algebra with Applications}. (CC-BY-NC-SA)

W. Keith Nicholson, {\it Linear Algebra with Applications}, Lyryx 2018, Open Edition, p 73-74.

\end{document}
