\documentclass{ximera}
%% You can put user macros here
%% However, you cannot make new environments

\listfiles

\graphicspath{
{./}
{./LTR-0070/}
{./VEC-0060/}
{./APP-0020/}
}

\usepackage{tikz}
\usepackage{tkz-euclide}
\usepackage{tikz-3dplot}
\usepackage{tikz-cd}
\usetikzlibrary{shapes.geometric}
\usetikzlibrary{arrows}
%\usetkzobj{all}
\pgfplotsset{compat=1.13} % prevents compile error.

%\renewcommand{\vec}[1]{\mathbf{#1}}
\renewcommand{\vec}{\mathbf}
\newcommand{\RR}{\mathbb{R}}
\newcommand{\dfn}{\textit}
\newcommand{\dotp}{\cdot}
\newcommand{\id}{\text{id}}
\newcommand\norm[1]{\left\lVert#1\right\rVert}
 
\newtheorem{general}{Generalization}
\newtheorem{initprob}{Exploration Problem}

\tikzstyle geometryDiagrams=[ultra thick,color=blue!50!black]

%\DefineVerbatimEnvironment{octave}{Verbatim}{numbers=left,frame=lines,label=Octave,labelposition=topline}



\usepackage{mathtools}


\title{A Brief Introduction to $\RR^n$} \license{CC BY-NC-SA 4.0}

\begin{document}

\begin{abstract}
\end{abstract}
\maketitle

\begin{onlineOnly}
\section*{A Brief Introduction to $\RR^n$}
\end{onlineOnly}

The set of all real numbers is denoted by $\RR$.  It is convenient to associate real numbers with points on a line, called the \dfn{real number line}.   

\begin{center}
\begin{tikzpicture}
\draw[latex-latex] (-3.5,0) -- (3.5,0) ; %edit here for the axis
\foreach \x in  {-3,-2,-1,0,1,2,3} % edit here for the vertical lines
\draw[shift={(\x,0)},color=black] (0pt,3pt) -- (0pt,-3pt);
\foreach \x in {-3,-2,-1,0,1,2,3} % edit here for the numbers
\draw[shift={(\x,0)},color=black] (0pt,0pt) -- (0pt,-3pt) node[below] 
{$\x$};
\end{tikzpicture}
\end{center}

The set of all ordered pairs $(x, y)$, where $x$ and $y$ are real numbers, is called $\RR^2$.  Using set notation we write:  
$$\RR^2=\{(x, y):x,y\in \RR\}$$
Geometrically speaking, $\RR^2$ can be associated with a coordinate plane in which we refer to each point by its $x$ and $y$ coordinates.
\begin{center}
\begin{tikzpicture}[line cap=round,line join=round,>=triangle 45,x=1cm,y=1cm]
\begin{axis}[
x=1cm,y=1cm,
axis lines=middle,
ymajorgrids=true,
xmajorgrids=true,
xmin=-4.5,
xmax=4.5,
ymin=-3.5,
ymax=3.5,
xtick={-4,-3,...,4},
ytick={-3,-2,...,3},]
\end{axis}
\end{tikzpicture}
\end{center}
The set of all ordered triples $(x, y, z)$, where $x$, $y$ and $z$ are real numbers,  is called $\RR^3$.  
$$\RR^3=\{(x, y, z):x,y, z\in \RR\}$$
Geometrically, an ordered triple of $\RR^3$ is associated with a point of a three-dimensional space whose position is given by  $x$, $y$ and $z$ coordinates.

\begin{center}
\tdplotsetmaincoords{70}{130}
\begin{tikzpicture}
	\draw[->](-2,0,0)--(5,0,0) node[below left]{$y$};
    \draw[->](0,-2,0)--(0,5,0) node[below left]{$z$};
    \draw[->](0,0,-2)--(0,0,5) node[below left]{$x$};
    
   \end{tikzpicture}
\end{center}

\begin{example}\label{ex:plotpointsr3} The following points are shown plotted in $\RR^3$.

  \begin{enumerate}
\item
$P(6, 8, 7)$
\item
$Q(4, -6, 9)$
\item
$R(-3, 9, -10)$
  \end{enumerate}
  
  
  \begin{center}
\begin{tikzpicture}[x=0.5cm,y=0.5cm,z=0.3cm]
% The axes
\draw[->] (xyz cs:x=-13.5) -- (xyz cs:x=13.5) node[above] {$y$};
\draw[->] (xyz cs:y=-13.5) -- (xyz cs:y=13.5) node[right] {$z$};
\draw[->] (xyz cs:z=13.5)--(xyz cs:z=-13.5) node[below] {$x$} ;
% The thin ticks
\foreach \coo in {-13,-12,...,13}
{
  \draw (\coo,-1.5pt) -- (\coo,1.5pt);
  \draw (-1.5pt,\coo) -- (1.5pt,\coo);
  \draw (xyz cs:y=-0.15pt,z=\coo) -- (xyz cs:y=0.15pt,z=\coo);
}

% Dashed lines for the points P
\draw[dashed,red] 
  (xyz cs:z=-6) -- 
  +(0,7) coordinate (u) -- 
  (xyz cs:y=7) -- 
  +(8,0) -- 
  ++(xyz cs:x=8,z=-6) coordinate (v) --
  +(0,-7) coordinate (w) --
  cycle;
\draw[dashed,red] (u) -- (v);
\draw[dashed,red] (8,7) -- (8,0) -- (w);

% Dots and labels for P
\node[fill,circle,inner sep=1.5pt,label={right:$P(6,8,7)$}] at (v) {};

% Dashed lines for the points Q
\draw[dashed,blue] 
  (xyz cs:z=-4) -- 
  +(0,9) coordinate (u1) -- 
  (xyz cs:y=9) -- 
  +(-6,0) -- 
  ++(xyz cs:x=-6,z=-4) coordinate (v1) --
  +(0,-9) coordinate (w1) --
  cycle;
\draw[dashed,blue] (u1) -- (v1);
\draw[dashed,blue] (-6,9) -- (-6,0) -- (w1);

% Dots and labels for Q
\node[fill,circle,inner sep=1.5pt,label={left:$Q(4,-6,9)$}] at (v1) {};

% Dashed lines for the points R
\draw[dashed] 
  (xyz cs:z=3) -- 
  +(0,-10) coordinate (u2) -- 
  (xyz cs:y=-10) -- 
  +(9,0) -- 
  ++(xyz cs:x=9,z=3) coordinate (v2) --
  +(0,10) coordinate (w2) --
  cycle;
\draw[dashed] (u2) -- (v2);
\draw[dashed] (9,-10) -- (9,0) -- (w2);

% Dots and labels for R
\node[fill,circle,inner sep=1.5pt,label={right:$R(-3,9,-10)$}] at (v2) {};

\end{tikzpicture}
\end{center}
\end{example}

Each pair of axes in $\RR^3$ determines a plane. The resulting three planes are called \dfn{coordinate planes}.  Each coordinate plane is named after the axes that determine it.  Thus, we have the $xy$-plane, $xz$-plane, and $yz$-plane.  Coordinate planes intersect at the point $(0, 0, 0)$, called the \dfn{origin}, and subdivide $\RR^3$ into eight regions, called \dfn{octants}. 

\begin{center}
\tdplotsetmaincoords{70}{130}
\begin{tikzpicture}
	\draw[->](-3,0,0)--(5,0,0) node[below left]{$y$};
    \draw[->](0,-2,0)--(0,5,0) node[below left]{$z$};
    \draw[->](0,0,-2)--(0,0,5) node[below left]{$x$};
       
    \filldraw[blue!30!white] (0,0,0)--(0,0,-2)--(4,0,-2)--(4,0,0)--cycle;
   \filldraw[blue!50!white] (0,-1,-2)--(0, 4, -2)--(0,4,0)--(0,-1,0)--cycle;
    \filldraw[blue!40!white] (-2,4,0)--(4,4,0)node[black, below left]{$yz$-plane}--(4,-1,0)--(-2,-1,0)--cycle;
    \filldraw[blue!30!white] (0,0,0)--(0,0,4)--(-2,0,4)--(-2,0,0)--cycle;
    \filldraw[blue!50!white] (0,-1,0)--(0, 4, 0)--(0,4,4)node[black, below right]{$xz$-plane}--(0,-1,4)--cycle;
    \filldraw[blue!30!white] (0,0,0)--(0,0,4)--(4,0,4)node[black, above left]{$xy$-plane}--(4,0,0)--cycle;
     
     \node[fill,circle,inner sep=1.5pt,label={right:$(0,0,0)$}] at (0,0,0) {};
\end{tikzpicture}
\end{center}



The set of all ordered $n$-tuples $(x_1, x_2, \ldots, x_n)$, where $x_i$ is a real number for $1\leq i\leq n$, is called $\RR^n$.
$$\RR^n=\{(x_1, x_2,\ldots,x_n)|x_i\in \RR\, \text{for}\, 1\leq i\leq n\}$$
The point $(0,0,\ldots, 0)$ in $\RR^n$ is called the \dfn{origin}. 

$\RR^n$ cannot be visualized for $n>3$, but many familiar ideas, such as the distance formula, can be generalized to $\RR^n$. 

\subsection*{Distance in $\RR^n$}
In this section we will establish a formula for the distance between two points in $\RR^n$.  We begin by observing that the distance between two numbers (points) $x_1$ and $x_2$ on the number line is given by $|x_1-x_2|$.  (Why do we use the absolute value brackets?).

We can use the Pythagorean Theorem to establish the distance formula for points of $\RR^2$.

\begin{center}
\begin{tikzpicture}[scale=1]
  \draw[line width=2pt,black](-1,0)--(5,0);
\draw[line width=2pt,black](0,-1)--(0,3);
    \draw[line width=1pt](1, 1)--(4,2.5);
 \fill (1,1)node[below right]{$A(x_1,y_1)$} circle (1.1mm);
\fill (4, 2.5)node[above]{$B(x_2,y_2)$} circle (1.1mm);
\draw[line width=1pt, dashed](1,1)--(1,0);
   \draw[line width=1pt, dashed](1,1)--(0,1);
   \draw[line width=1pt, dashed](4,2.5)--(4,0);
   \draw[line width=1pt, dashed](4, 2.5)--(0,2.5);
   \filldraw[blue, opacity=0.3](1,1)--(4,1)--(4,2.5)--cycle;
    \node[black] at (2.5, -0.4)   (a) {$|x_1-x_2|$};
    \node[black] at (-0.9, 1.7)   (a) {$|y_1-y_2|$};
 \end{tikzpicture}
\end{center}

Let $A(x_1, y_1)$ and $B(x_2, y_2)$ be points in $\RR^2$.  By the Pythagorean Theorem we have
$$AB^2=(x_1-x_2)^2+(y_1-y_2)^2$$
$$AB=\sqrt{(x_1-x_2)^2+(y_1-y_2)^2}$$
Why were we able to drop the absolute value brackets?

The distance formula for points in $\RR^3$ can also be derived using the Pythagorean Theorem.  Let $A(x_1, y_1, z_1)$ and $B(x_2, y_2, z_2)$ be points of $\RR^3$.  Use the navigation bar in the following GeoGebra interactive to walk through the steps of the derivation of the distance formula.

\pdfOnly{
Access GeoGebra interactives through the online version of this text at 

\href{https://ximera.osu.edu/oerlinalg}{https://ximera.osu.edu/oerlinalg}.
}

\begin{onlineOnly}
\begin{center} 
\geogebra{dc267r6v}{800}{600} 
\end{center}
\end{onlineOnly}

\begin{formula}\label{form:distR3}
Let $A(x_1, y_1, z_1)$ and $B(x_2, y_2, z_2)$ be points in $\RR^3$.  The distance between $A$ and $B$ is given by
$$AB=\sqrt{(x_1-x_2)^2+(y_1-y_2)^2+(z_1-z_2)^2}$$
\end{formula}

Observe the similarity of pattern in the distance formulas for $\RR^1$, $\RR^2$ and $\RR^3$.  We will take advantage of this pattern to define the distance between two points of $\RR^n$.

\begin{formula}\label{form:distRn}
Let $A(a_1, a_2,\ldots ,a_n)$ and $B(b_1, b_2,\ldots ,b_n)$ be points in $\RR^n$.  The distance between $A$ and $B$ is given by
$$AB=\sqrt{(a_1-b_1)^2+(a_2-b_2)^2+\ldots +(a_n-b_n)^2}$$
\end{formula}



\section*{Practice Problems}
\begin{problem}\label{prob:plotpointsr3}
Find the coordinates of each point.

  \begin{center}
\begin{tikzpicture}[x=0.5cm,y=0.5cm,z=0.3cm]
% The axes
\draw[->] (xyz cs:x=-13.5) -- (xyz cs:x=13.5) node[above] {$y$};
\draw[->] (xyz cs:y=-13.5) -- (xyz cs:y=13.5) node[right] {$z$};
\draw[->] (xyz cs:z=13.5)--(xyz cs:z=-13.5) node[below] {$x$} ;
% The thin ticks
\foreach \coo in {-13,-12,...,13}
{
  \draw (\coo,-1.5pt) -- (\coo,1.5pt);
  \draw (-1.5pt,\coo) -- (1.5pt,\coo);
  \draw (xyz cs:y=-0.15pt,z=\coo) -- (xyz cs:y=0.15pt,z=\coo);
}

% Dashed lines for the points P
\draw[dashed,red] 
  (xyz cs:z=-5) -- 
  +(0,10) coordinate (u) -- 
  (xyz cs:y=10) -- 
  +(5,0) -- 
  ++(xyz cs:x=5,z=-5) coordinate (v) --
  +(0,-10) coordinate (w) --
  cycle;
\draw[dashed,red] (u) -- (v);
\draw[dashed,red] (5,10) -- (5,0) -- (w);

% Dots and labels for P
\node[fill,circle,inner sep=1.5pt,label={right:$P$}] at (v) {};

% Dashed lines for the points Q
\draw[dashed,blue] 
  (xyz cs:z=6) -- 
  +(0,2) coordinate (u1) -- 
  (xyz cs:y=2) -- 
  +(3,0) -- 
  ++(xyz cs:x=3,z=6) coordinate (v1) --
  +(0,-2) coordinate (w1) --
  cycle;
\draw[dashed,blue] (u1) -- (v1);
\draw[dashed,blue] (3,2) -- (3,0) -- (w1);

% Dots and labels for Q
\node[fill,circle,inner sep=1.5pt,label={right:$Q$}] at (v1) {};

% Dashed lines for the points R
\draw[dashed] 
  (xyz cs:z=-3) -- 
  +(0,-8) coordinate (u2) -- 
  (xyz cs:y=-8) -- 
  +(-5,0) -- 
  ++(xyz cs:x=-5,z=-3) coordinate (v2) --
  +(0,8) coordinate (w2) --
  cycle;
\draw[dashed] (u2) -- (v2);
\draw[dashed] (-5,-8) -- (-5,0) -- (w2);

% Dots and labels for R
\node[fill,circle,inner sep=1.5pt,label={left:$R$}] at (v2) {};

\end{tikzpicture}
\end{center}
Answer:
  $$P(\answer{5}, \answer{5}, \answer{10})$$
 
  $$Q(\answer{-6}, \answer{3}, \answer{2})$$

  $$R(\answer{3}, \answer{-5}, \answer{-8})$$

\end{problem}

\begin{problem}\label{prob:geogebrapts}
Find the coordinates of each point.
RIGHT-CLICK and DRAG to rotate the image.

\pdfOnly{
Access GeoGebra interactives through the online version of this text at 

\href{https://ximera.osu.edu/oerlinalg}{https://ximera.osu.edu/oerlinalg}.
}

\begin{onlineOnly}
\begin{center} 
\geogebra{bynu3r84}{800}{600} 
\end{center}
\end{onlineOnly}

Answer:
  $$A(\answer{-1}, \answer{4}, \answer{2})$$
 
  $$B(\answer{3}, \answer{2}, \answer{-3})$$

  $$C(\answer{2}, \answer{-5}, \answer{3})$$
  
  $$D(\answer{-3}, \answer{-2}, \answer{4})$$

\end{problem}

\begin{problem}\label{prob:distR3}
Find the distance between $A(-2, -1, 4)$ and $B(1, -5, -8)$.
$$AB=\answer{13}$$
\end{problem}

\begin{problem}\label{prob:distR4}
Consider the equation 
$$x^2+y^2+z^2+w^2=25$$

What can be said about all points $(x, y, z, w)\in \RR^4$ that satisfy this equation?

\begin{multipleChoice}
 \choice{Such points are equidistant from the origin.}
 \choice{Such points form a four-dimensional sphere of radius $5$.}
 \choice{Such points are located 5 units from the origin.}
 \choice[correct]{All of the above.}
  \end{multipleChoice}


\end{problem}

\end{document} 