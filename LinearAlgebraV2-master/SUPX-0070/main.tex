\documentclass{ximera}
%% You can put user macros here
%% However, you cannot make new environments

\listfiles

\graphicspath{
{./}
{./LTR-0070/}
{./VEC-0060/}
{./APP-0020/}
}

\usepackage{tikz}
\usepackage{tkz-euclide}
\usepackage{tikz-3dplot}
\usepackage{tikz-cd}
\usetikzlibrary{shapes.geometric}
\usetikzlibrary{arrows}
%\usetkzobj{all}
\pgfplotsset{compat=1.13} % prevents compile error.

%\renewcommand{\vec}[1]{\mathbf{#1}}
\renewcommand{\vec}{\mathbf}
\newcommand{\RR}{\mathbb{R}}
\newcommand{\dfn}{\textit}
\newcommand{\dotp}{\cdot}
\newcommand{\id}{\text{id}}
\newcommand\norm[1]{\left\lVert#1\right\rVert}
 
\newtheorem{general}{Generalization}
\newtheorem{initprob}{Exploration Problem}

\tikzstyle geometryDiagrams=[ultra thick,color=blue!50!black]

%\DefineVerbatimEnvironment{octave}{Verbatim}{numbers=left,frame=lines,label=Octave,labelposition=topline}



\usepackage{mathtools}


\title{Additional Exercises for Ch 7} \license{CC BY-NC-SA 4.0}

\begin{document}

\begin{abstract}
\end{abstract}
\maketitle

\section*{Additional Exercises for Chapter 7: Determinants}

\begin{problem}\label{prb:7.1} Find the determinants of the following matrices.

\begin{enumerate}
\item $\left[
\begin{array}{rr}
1 & 3 \\
0 & 2
\end{array}
\right]$

\item $\left[
\begin{array}{rr}
0 & 3 \\
0 & 2
\end{array}
\right]$

\item $\left[
\begin{array}{rr}
4 & 3 \\
6 & 2
\end{array}
\right]$
\end{enumerate}
%\begin{hint}
%\begin{enumerate}
%\item
%\end{enumerate}
%\end{hint}
\end{problem}

\begin{problem}\label{prb:7.2} Let $A = \left[ \begin{array}{rrr}
1 & 2 & 4 \\
0 & 1 & 3 \\
-2 & 5 & 1
\end{array} \right]$. When doing cofactor expansion along the top row, we encounter three minor matrices.  What are they?

Click the arrow to see answer.
\begin{expandable}
$$\begin{bmatrix}1&3\\5&1\end{bmatrix},\quad\begin{bmatrix}0&3\\-2&1\end{bmatrix},\quad\begin{bmatrix}0&1\\-2&5\end{bmatrix}$$
\end{expandable}
\end{problem}

\begin{problem}\label{prb:7.3} Find the determinants of the following matrices.
\begin{enumerate}
\item $\left[
\begin{array}{rrr}
1 & 2 & 3 \\
3 & 2 & 2 \\
0 & 9 & 8
\end{array}
\right] $
\item $\left[
\begin{array}{rrr}
4 & 3 & 2 \\
1 & 7 & 8 \\
3 & -9 & 3
\end{array}
\right] $
\item $\left[
\begin{array}{rrrr}
1 & 2 & 3 & 2 \\
1 & 3 & 2 & 3 \\
4 & 1 & 5 & 0 \\
1 & 2 & 1 & 2
\end{array}
\right] $
\end{enumerate}

Click the arrow to see answer.
\begin{expandable}
\begin{enumerate}
\item The answer is $31$.
\item The answer is $375$.
\item The answer is $-2$.
\end{enumerate}
\end{expandable}
\end{problem}

\begin{problem}\label{prb:7.4} Find the following determinant by (a) expanding along the first row, (b)
second column.
\begin{equation*}
\left|
\begin{array}{rrr}
1 & 2 & 1 \\
2 & 1 & 3 \\
2 & 1 & 1
\end{array}
\right|
\end{equation*}

Click the arrow to see answer.
\begin{expandable}
\[
\left|
\begin{array}{ccc}
1 & 2 & 1 \\
2 & 1 & 3 \\
2 & 1 & 1
\end{array}
\right| =  6
\]
\end{expandable}
\end{problem}

\begin{problem}\label{prb:7.5} Find the following determinant by (a) expanding along the first column, (b)
third row.
\begin{equation*}
\left|
\begin{array}{rrr}
1 & 2 & 1 \\
1 & 0 & 1 \\
2 & 1 & 1
\end{array}
\right|
\end{equation*}

Click the arrow to see answer.
\begin{expandable}
\[
\left|
\begin{array}{ccc}
1 & 2 & 1 \\
1 & 0 & 1 \\
2 & 1 & 1
\end{array}
\right| =  2
\]
\end{expandable}
\end{problem}

\begin{problem}\label{prb:7.6} Find the following determinant by expanding (a) along the second row, (b)
first column.
\begin{equation*}
\left|
\begin{array}{rrr}
1 & 2 & 1 \\
2 & 1 & 3 \\
2 & 1 & 1
\end{array}
\right|
\end{equation*}

Click the arrow to see answer.
\begin{expandable}
\[
\left|
\begin{array}{ccc}
1 & 2 & 1 \\
2 & 1 & 3 \\
2 & 1 & 1
\end{array}
\right| = 6
\]
\end{expandable}
\end{problem}

\begin{problem}\label{prb:7.7} Compute the determinant by cofactor expansion. Pick the easiest row or
column to use.
\begin{equation*}
\left|
\begin{array}{rrrr}
1 & 0 & 0 & 1 \\
2 & 1 & 1 & 0 \\
0 & 0 & 0 & 2 \\
2 & 1 & 3 & 1
\end{array}
\right|
\end{equation*}

Click the arrow to see answer.
\begin{expandable}
\[
\left|
\begin{array}{cccc}
1 & 0 & 0 & 1 \\
2 & 1 & 1 & 0 \\
0 & 0 & 0 & 2 \\
2 & 1 & 3 & 1
\end{array}
\right| = -4
\]
\end{expandable}
\end{problem}

\begin{problem}\label{prb:7.8} Find the determinants of the following matrices.

\begin{enumerate}
\item
$A = \left[ \begin{array}{rr}
1 & -34 \\
0 & 2
\end{array} \right] $

\item
$A = \left[ \begin{array}{rrr}
4 & 3 & 14 \\
 0 & -2 & 0 \\
0 & 0 & 5
\end{array} \right]$

\item
$A = \left[ \begin{array}{rrrr}
2 & 3 & 15 & 0 \\
0 & 4 & 1 & 7 \\
0 & 0 & -3 & 5 \\
0 & 0 & 0 & 1
\end{array} \right]$
\end{enumerate}

%\begin{hint}
%\begin{enumerate}
%\item
%\end{enumerate}
%\end{hint}

\end{problem}

\begin{problem}\label{prb:7.9} An operation is done to get from the first matrix to the second.
Identify what was done and tell how it will affect the value of the
determinant.
\begin{equation*}
\left[
\begin{array}{cc}
a & b \\
c & d
\end{array}
\right]  \rightarrow \cdots \rightarrow \left[
\begin{array}{cc}
a & c \\
b & d
\end{array}
\right]
\end{equation*}

Click the arrow to see answer.
\begin{expandable}
It does not change the determinant. This was just taking the transpose.
\end{expandable}
\end{problem}

\begin{problem}\label{prb:7.10} An operation is done to get from the first matrix to the second.
Identify what was done and tell how it will affect the value of the
determinant.
\begin{equation*}
\left[
\begin{array}{cc}
a & b \\
c & d
\end{array}
\right] \rightarrow \cdots \rightarrow \left[
\begin{array}{cc}
c & d \\
a & b
\end{array}
\right]
\end{equation*}

Click the arrow to see answer.
\begin{expandable}
In this case two rows were switched and so the resulting determinant is $-1$
times the first.
\end{expandable}
\end{problem}


\begin{problem}\label{prb:7.11} An operation is done to get from the first matrix to the second.
Identify what was done and tell how it will affect the value of the
determinant.
\begin{equation*}
\left[
\begin{array}{cc}
a & b \\
c & d
\end{array}
\right] \rightarrow \cdots \rightarrow \left[
\begin{array}{cc}
a & b \\
a+c & b+d
\end{array}
\right]
\end{equation*}

Click the arrow to see answer.
\begin{expandable}
The determinant is unchanged. It was just the first row added to the second.
\end{expandable}
\end{problem}


\begin{problem}\label{prb:7.12} An operation is done to get from the first matrix to the second.
Identify what was done and tell how it will affect the value of the
determinant.
\begin{equation*}
\left[
\begin{array}{cc}
a & b \\
c & d
\end{array}
\right] \rightarrow \cdots \rightarrow \left[
\begin{array}{cc}
a & b \\
2c & 2d
\end{array}
\right]
\end{equation*}

Click the arrow to see answer.
\begin{expandable}
The second row was multiplied by 2 so the determinant of the result is 2
times the original determinant.
\end{expandable}
\end{problem}

\begin{problem}\label{prb:7.13} An operation is done to get from the first matrix to the second.
Identify what was done and tell how it will affect the value of the
determinant.
\begin{equation*}
\left[
\begin{array}{cc}
a & b \\
c & d
\end{array}
\right] \rightarrow \cdots \rightarrow \left[
\begin{array}{cc}
b & a \\
d & c
\end{array}
\right]
\end{equation*}

Click the arrow to see answer.
\begin{expandable}
In this case the two columns were switched so the determinant of the second
is $-1$ times the determinant of the first.
\end{expandable}
\end{problem}


\begin{problem}\label{prb:7.14} Let $A$ be an $r\times r$ matrix and suppose there are $r-1$ rows
(columns) such that all rows (columns) are linear combinations of these $r-1$
rows (columns). Show $\det \left( A\right) =0.$

\begin{hint}
If the determinant is nonzero, then it will remain nonzero with row operations applied to the matrix.
In this case, you can obtain a row of zeros by doing row
operations. Thus the determinant must be zero.
\end{hint}
\end{problem}

\begin{problem}\label{prb:7.15} Show $\det \left( aA\right) =a^{n}\det \left( A\right) $ for an $n \times n $ matrix $A
$ and scalar $a$.

Click the arrow to see answer.
\begin{expandable}
$\det \left( aA\right) =\det
\left( aIA\right) =\det \left( aI\right) \det \left( A\right) =a^{n}\det
\left( A\right) .$ The matrix which has $a$ down the main diagonal has
determinant equal to $a^{n}$.
\end{expandable}
\end{problem}


\begin{problem}\label{prb:7.16} Construct $2\times 2$ matrices $A$ and $B$ to show that the
$\det A \det B = \det (AB)$.

Click the arrow to see answer.
\begin{expandable}
\[
\det
\left( \left[
\begin{array}{cc}
1 & 2 \\
3 & 4
\end{array}
\right] \left[
\begin{array}{rr}
-1 & 2 \\
-5 & 6
\end{array}
\right] \right) = -8
\]
\[
\det \left[
\begin{array}{cc}
1 & 2 \\
3 & 4
\end{array}
\right] \det \left[
\begin{array}{rr}
-1 & 2 \\
-5 & 6
\end{array}
\right] = -2 \times 4 = -8
\]
\end{expandable}
\end{problem}

\begin{problem}\label{prb:7.17} Is it true that $\det \left( A+B\right) =\det \left( A\right) +\det
\left( B\right) ?$ If this is so, explain why. If it is not so,
give a counter example.

Click the arrow to see answer.
\begin{expandable}
This is not true at all. Consider $A=\left[
\begin{array}{cc}
1 & 0 \\
0 & 1
\end{array}
\right] ,B=\left[
\begin{array}{rr}
-1 & 0 \\
0 & -1
\end{array}
\right] .$
\end{expandable}
\end{problem}

\begin{problem}\label{prb:7.18} An $n\times n$ matrix is called \dfn{nilpotent}
if for some positive integer, $k$ it follows $A^{k}=O.$ If
$A$ is a nilpotent matrix and $k$ is the smallest possible integer such that
$A^{k}=O,$ what are the possible values of $\det \left( A\right)$?

Click the arrow to see answer.
\begin{expandable}
It must
be $0$ because $0=\det \left( 0\right) =\det \left( A^{k}\right) =\left( \det
\left( A\right) \right) ^{k}.$
\end{expandable}
\end{problem}

\begin{problem}\label{prb:7.19}A matrix is said to be \dfn{orthogonal} if
$A^{T}A=I.$ Thus the inverse of an orthogonal matrix is just its transpose.
What are the possible values of $\det \left( A\right) $ if $A$ is an
orthogonal matrix?

Click the arrow to see answer.
\begin{expandable}
You would need $\det \left( AA^{T}\right) =\det
\left( A\right) \det \left( A^{T}\right) =\det \left( A\right) ^{2}=1$ and
so $\det \left( A\right) =1,$ or $-1$.
\end{expandable}
\end{problem}

\begin{problem}\label{prb:7.20} Let $A$ and $B$ be two $n\times n$ matrices. $A\sim B$
($A$ is \textbf{similar} to $B$) means there exists an invertible matrix $P$
such that $A=P^{-1}BP.$ Show that if $A\sim B,$ then
$\det \left( A\right) =\det \left( B\right)$.

Click the arrow to see answer.
\begin{expandable}
$\det \left( A\right) =\det
\left( S^{-1}BS\right) =\det \left( S^{-1}\right) \det \left( B\right) \det
\left( S\right) =\det \left( B\right) \det \left( S^{-1}S\right) =\det
\left( B\right) $.
\end{expandable}
\end{problem}

\begin{problem}\label{prb:7.21} Tell whether each statement is true or false. If true, provide a proof. If false, provide a counter example.
\begin{enumerate}
\item If $A$ is a $3\times 3$ matrix with a zero determinant, then one
column must be a multiple of some other column.

\item If any two columns of a square matrix are equal, then the determinant
of the matrix equals zero.

\item For two $n\times n$ matrices $A$ and $B$, $\det \left( A+B\right)
=\det \left( A\right) +\det \left( B\right) .$

\item For an $n\times n$ matrix $A$, $\det \left( 3A\right) =3\det \left(
A\right) $

\item If $A^{-1}$ exists then $\det \left( A^{-1}\right) =\det \left(
A\right) ^{-1}.$

\item If $B$ is obtained by multiplying a single row of $A$ by $4$ then $%
\det \left( B\right) =4\det \left( A\right) .$

\item For $A$ an $n\times n$ matrix, $\det \left( -A\right) =\left(
-1\right) ^{n}\det \left( A\right) .$

\item If $A$ is a real $n\times n$ matrix, then $\det \left( A^{T}A\right)
\geq 0.$

\item If $A^{k}=O$ for some positive integer $k,$ then $\det \left(
A\right) =0.$

\item If $AX=0$ for some $X \neq 0,$ then $\det \left(
A\right) =0.$
\end{enumerate}

Click the arrow to see answer.
\begin{expandable}
\begin{enumerate}
\item False. Consider $\left[
\begin{array}{rrr}
1 & 1 & 2 \\
-1 & 5 & 4 \\
0 & 3 & 3
\end{array}
\right] $
\item True.
\item False.
\item False.
\item True.
\item True.
\item True.
\item True.
\item True.
\item True.
\end{enumerate}
\end{expandable}
\end{problem}

\begin{problem}\label{prb:7.22} Find the determinant using row operations to first simplify.
\begin{equation*}
\left|
\begin{array}{rrr}
1 & 2 & 1 \\
2 & 3 & 2 \\
-4 & 1 & 2
\end{array}
\right|
\end{equation*}

Click the arrow to see answer.
\begin{expandable}
\[
\left|
\begin{array}{rrr}
1 & 2 & 1 \\
2 & 3 & 2 \\
-4 & 1 & 2
\end{array}
\right| = -6
\]
\end{expandable}
\end{problem}

\begin{problem}\label{prb:7.23} Find the determinant using row operations to first simplify.
\begin{equation*}
\left|
\begin{array}{rrr}
2 & 1 & 3 \\
2 & 4 & 2 \\
1 & 4 & -5
\end{array}
\right|
\end{equation*}

Click the arrow to see answer.
\begin{expandable}
\[
\left|
\begin{array}{rrr}
2 & 1 & 3 \\
2 & 4 & 2 \\
1 & 4 & -5
\end{array}
\right| = -32
\]
\end{expandable}
\end{problem}

\begin{problem}\label{prb:7.24} Find the determinant using row operations to first simplify.
\begin{equation*}
\left|
\begin{array}{rrrr}
1 & 2 & 1 & 2 \\
3 & 1 & -2 & 3 \\
-1 & 0 & 3 & 1 \\
2 & 3 & 2 & -2
\end{array}
\right|
\end{equation*}

Click the arrow to see answer.
\begin{expandable}
One can row reduce this using only row operation 3 to
\[
\left[
\begin{array}{rrrr}
1 & 2 & 1 & 2 \\
0 & -5 & -5 & -3 \\
0 & 0 & 2 & \frac{9}{5} \\
0 & 0 & 0 & -\frac{63}{10}
\end{array}
\right]
\]
and therefore, the determinant is $-63.$
\[
\left|
\begin{array}{rrrr}
1 & 2 & 1 & 2 \\
3 & 1 & -2 & 3 \\
-1 & 0 & 3 & 1 \\
2 & 3 & 2 & -2
\end{array}
\right| = 63
\]
\end{expandable}
\end{problem}

\begin{problem}\label{prb:7.25} Find the determinant using row operations to first simplify.
\begin{equation*}
\left|
\begin{array}{rrrr}
1 & 4 & 1 & 2 \\
3 & 2 & -2 & 3 \\
-1 & 0 & 3 & 3 \\
2 & 1 & 2 & -2
\end{array}
\right|
\end{equation*}

Click the arrow to see answer.
\begin{expandable}
One can row reduce this using only row operation 3 to

\[
\left[
\begin{array}{rrrr}
1 & 4 & 1 & 2 \\
0 & -10 & -5 & -3 \\
0 & 0 & 2 & \frac{19}{5} \\
0 & 0 & 0 & -\frac{211}{20}
\end{array}
\right]
\]
Thus the determinant is given by
\[
\left|
\begin{array}{rrrr}
1 & 4 & 1 & 2 \\
3 & 2 & -2 & 3 \\
-1 & 0 & 3 & 3 \\
2 & 1 & 2 & -2
\end{array}
\right| = 211
\]
\end{expandable}
\end{problem}

\begin{problem}\label{prb:7.26} Let
\begin{equation*}
A=
\left[
\begin{array}{rrr}
1 & 2 & 3 \\
0 & 2 & 1 \\
3 & 1 & 0
\end{array}
\right]
\end{equation*}
Determine whether the matrix $A$ has an inverse by finding whether the
determinant is non zero. If the determinant is nonzero, find the inverse
using the formula for the inverse which involves the cofactor matrix.

Click the arrow to see answer.
\begin{expandable}
$\det
\left[
\begin{array}{ccc}
1 & 2 & 3 \\
0 & 2 & 1 \\
3 & 1 & 0
\end{array}
\right] = -13$ and so it has an inverse. This inverse is
\begin{eqnarray*}
\frac{1}{-13}\left[
\begin{array}{rrr}
\left\vert
\begin{array}{cc}
2 & 1 \\
1 & 0
\end{array}
\right\vert  & -\left\vert
\begin{array}{cc}
0 & 1 \\
3 & 0
\end{array}
\right\vert  & \left\vert
\begin{array}{cc}
0 & 2 \\
3 & 1
\end{array}
\right\vert  \\
-\left\vert
\begin{array}{cc}
2 & 3 \\
1 & 0
\end{array}
\right\vert  & \left\vert
\begin{array}{cc}
1 & 3 \\
3 & 0
\end{array}
\right\vert  & -\left\vert
\begin{array}{cc}
1 & 2 \\
3 & 1
\end{array}
\right\vert  \\
\left\vert
\begin{array}{cc}
2 & 3 \\
2 & 1
\end{array}
\right\vert  & -\left\vert
\begin{array}{cc}
1 & 3 \\
0 & 1
\end{array}
\right\vert  & \left\vert
\begin{array}{cc}
1 & 2 \\
0 & 2
\end{array}
\right\vert
\end{array}
\right] ^{T} &=&\frac{1}{-13}\left[
\begin{array}{rrr}
-1 & 3 & -6 \\
3 & -9 & 5 \\
-4 & -1 & 2
\end{array}
\right] ^{T} \\
&=& \left[
\begin{array}{rrr}
\frac{1}{13} & -\frac{3}{13} & \frac{4}{13} \\
-\frac{3}{13} & \frac{9}{13} & \frac{1}{13} \\
\frac{6}{13} & -\frac{5}{13} & -\frac{2}{13}
\end{array}
\right]
\end{eqnarray*}
\end{expandable}
\end{problem}

\begin{problem}\label{prb:7.27} Let
\begin{equation*}
A=
\left[
\begin{array}{rrr}
1 & 2 & 0 \\
0 & 2 & 1 \\
3 & 1 & 1
\end{array}
\right]
\end{equation*}
Determine whether the matrix $A$ has an inverse by finding whether the
determinant is non zero. If the determinant is nonzero, find the inverse
using the formula for the inverse.

Click the arrow to see answer.
\begin{expandable}
$\det
\left[
\begin{array}{ccc}
1 & 2 & 0 \\
0 & 2 & 1 \\
3 & 1 & 1
\end{array}
\right] = 7$ so it has an inverse. This inverse is $\frac{1}{7}
\left[
\begin{array}{rrr}
1 & 3 & -6 \\
-2 & 1 & 5 \\
2 & -1 & 2
\end{array}
\right]^{T} = \left[
\begin{array}{rrr}
\frac{1}{7} & -\frac{2}{7} & \frac{2}{7} \\
\frac{3}{7} & \frac{1}{7} & -\frac{1}{7} \\
-\frac{6}{7} & \frac{5}{7} & \frac{2}{7}
\end{array}
\right] $
\end{expandable}
\end{problem}

\begin{problem}\label{prb:7.28} Let
\begin{equation*}
A=
\left[
\begin{array}{rrr}
1 & 3 & 3 \\
2 & 4 & 1 \\
0 & 1 & 1
\end{array}
\right]
\end{equation*}
Determine whether the matrix $A$ has an inverse by finding whether the
determinant is non zero. If the determinant is nonzero, find the inverse
using the formula for the inverse.

Click the arrow to see answer.
\begin{expandable}
\[
\det \left[
\begin{array}{ccc}
1 & 3 & 3 \\
2 & 4 & 1 \\
0 & 1 & 1
\end{array}
\right] = 3
\]
so it has an inverse which is
\[
\left[
\begin{array}{rrr}
1 & 0 & -3 \\
-\frac{2}{3} & \frac{1}{3} & \frac{5}{3} \\
\frac{2}{3} & -\frac{1}{3} & -\frac{2}{3}
\end{array}
\right]
\]
\end{expandable}
\end{problem}

\begin{problem}\label{prb:7.29} Let
\begin{equation*}
A =
\left[
\begin{array}{rrr}
1 & 2 & 3 \\
0 & 2 & 1 \\
2 & 6 & 7
\end{array}
\right]
\end{equation*}
Determine whether the matrix $A$ has an inverse by finding whether the
determinant is non zero. If the determinant is nonzero, find the inverse
using the formula for the inverse.
%\begin{hint}
%\end{hint}
\end{problem}


\begin{problem}\label{prb:7.30} Let
\begin{equation*}
A =
\left[
\begin{array}{rrr}
1 & 0 & 3 \\
1 & 0 & 1 \\
3 & 1 & 0
\end{array}
\right]
\end{equation*}
Determine whether the matrix $A$ has an inverse by finding whether the
determinant is non zero. If the determinant is nonzero, find the inverse
using the formula for the inverse.

Click the arrow to see answer.
\begin{expandable}
\[
\det \left[
\begin{array}{rrr}
1 & 0 & 3 \\
1 & 0 & 1 \\
3 & 1 & 0
\end{array}
\right] = 2
\]
and so it has an inverse. The inverse turns out to equal
\[
\left[
\begin{array}{rrr}
-\frac{1}{2} & \frac{3}{2} & 0 \\
\frac{3}{2} & -\frac{9}{2} & 1 \\
\frac{1}{2} & -\frac{1}{2} & 0
\end{array}
\right]
\]
\end{expandable}
\end{problem}

\begin{problem}\label{prb:7.31} For the following matrices, determine if they are invertible. If so, use the formula for the inverse in terms of the cofactor matrix to
find each inverse. If the inverse does not exist, explain why.
\begin{enumerate}
\item
$\left[
\begin{array}{rr}
1 & 1 \\
1 & 2
\end{array}
\right]$
\item
$\left[
\begin{array}{rrr}
1 & 2 & 3 \\
0 & 2 & 1 \\
4 & 1 & 1
\end{array}
\right]$
\item
$\left[
\begin{array}{rrr}
1 & 2 & 1 \\
2 & 3 & 0 \\
0 & 1 & 2
\end{array}
\right] $
\end{enumerate}
\begin{hint}
\begin{enumerate}
\item $\left\vert
\begin{array}{cc}
1 & 1 \\
1 & 2
\end{array}
\right\vert = 1$
\item $\left\vert
\begin{array}{ccc}
1 & 2 & 3 \\
0 & 2 & 1 \\
4 & 1 & 1%
\end{array}
\right\vert = -15$
\item $\left\vert
\begin{array}{ccc}
1 & 2 & 1 \\
2 & 3 & 0 \\
0 & 1 & 2
\end{array}
\right\vert = 0$
\end{enumerate}
\end{hint}
\end{problem}

\subsection*{Challenge Exercises}

\begin{problem}\label{prb:7.32} Consider the matrix
\begin{equation*}
A =
\left[
\begin{array}{ccc}
1 & 0 & 0 \\
0 & \cos t & -\sin t \\
0 & \sin t & \cos t
\end{array}
\right]
\end{equation*}
Does there exist a value of $t$ for which this matrix fails to have an
inverse? Explain.

Click the arrow to see answer.
\begin{expandable}
 No. It has a nonzero determinant for all $t$
\end{expandable}
\end{problem}


\begin{problem}\label{prb:7.33} Consider the matrix
\begin{equation*}
A =
\left[
\begin{array}{rrr}
1 & t & t^{2} \\
0 & 1 & 2t \\
t & 0 & 2
\end{array}
\right]
\end{equation*}
Does there exist a value of $t$ for which this matrix fails to have an
inverse? Explain.

Click the arrow to see answer.
\begin{expandable}
\[
\det \left[
\begin{array}{ccc}
1 & t & t^{2} \\
0 & 1 & 2t \\
t & 0 & 2
\end{array}
\right] = t^{3}+2
\]
and so it has no inverse when $t=-\sqrt[3]{2}$
\end{expandable}
\end{problem}

\begin{problem}\label{prb:7.34} Consider the matrix
\begin{equation*}
A =
\left[
\begin{array}{ccc}
e^{t} & \cosh t & \sinh t \\
e^{t} & \sinh t & \cosh t \\
e^{t} & \cosh t & \sinh t
\end{array}
\right]
\end{equation*}
Does there exist a value of $t$ for which this matrix fails to have an
inverse? Explain.

Click the arrow to see answer.
\begin{expandable}
\[
\det \left[
\begin{array}{ccc}
e^{t} & \cosh t & \sinh t \\
e^{t} & \sinh t & \cosh t \\
e^{t} & \cosh t & \sinh t
\end{array}
\right] = 0
\]
and so this matrix fails to have a nonzero determinant at any value of $t$.
\end{expandable}
\end{problem}

\begin{problem}\label{prb:7.35} Consider the matrix
\begin{equation*}
A =
\left[
\begin{array}{ccc}
e^{t} & e^{-t}\cos t & e^{-t}\sin t \\
e^{t} & -e^{-t}\cos t-e^{-t}\sin t & -e^{-t}\sin t+e^{-t}\cos t \\
e^{t} & 2e^{-t}\sin t & -2e^{-t}\cos t
\end{array}
\right]
\end{equation*}
Does there exist a value of $t$ for which this matrix fails to have an
inverse? Explain.

Click the arrow to see answer.
\begin{expandable}
\[
\det \left[
\begin{array}{ccc}
e^{t} & e^{-t}\cos t & e^{-t}\sin t \\
e^{t} & -e^{-t}\cos t-e^{-t}\sin t & -e^{-t}\sin t+e^{-t}\cos t \\
e^{t} & 2e^{-t}\sin t & -2e^{-t}\cos t%
\end{array}
\right] = 5e^{-t} \neq 0
\]
and so this matrix is always invertible.
\end{expandable}
\end{problem}

\begin{problem}\label{prb:7.36} Show that if $\det \left( A\right) \neq 0$ for $A$
an $n\times n$ matrix, it follows that if $AX=0,$ then $X=0$.

\begin{hint}
If $\det \left( A\right) \neq 0,$ then $A^{-1}$ exists and so you could
multiply on both sides on the left by $A^{-1}$ and obtain that $X=0$.
\end{hint}
\end{problem}

\begin{problem}\label{prb:7.37} Suppose $A,B$ are $n\times n$ matrices and that $AB=I.$ Show that then
$BA=I.$ 
\begin{hint}
First explain why
$\det \left( A\right) ,\det \left( B\right) $ are both nonzero. Then $\left(
AB\right) A=A$ and then show $BA\left( BA-I\right) =0.$ From this use what
is given to conclude $A\left( BA-I\right) =0.$ Then use Problem
\ref{prb:7.36}.
\end{hint}

Click the arrow to see answer.
\begin{expandable}
You have $1=\det \left( A\right) \det \left( B\right) $.
Hence both $A$ and $B$ have inverses. Letting $X$ be given,
\[
A\left( BA-I\right) X=\left( AB\right) AX-AX=AX-AX = 0
\]
and so it follows from the above problem that $\left( BA-I\right)X=0.$ Since $X$ is arbitrary, it follows that $BA=I.$
\end{expandable}
\end{problem}

\begin{problem}\label{prb:7.38} Use the formula for the inverse in terms of the cofactor matrix to
find the inverse of the matrix
\begin{equation*}
A=\left[
\begin{array}{ccc}
e^{t} & 0 & 0 \\
0 & e^{t}\cos t & e^{t}\sin t \\
0 & e^{t}\cos t-e^{t}\sin t & e^{t}\cos t+e^{t}\sin t
\end{array}
\right]
\end{equation*}

Click the arrow to see answer.
\begin{expandable}
\[
\det \left[
\begin{array}{ccc}
e^{t} & 0 & 0 \\
0 & e^{t}\cos t & e^{t}\sin t \\
0 & e^{t}\cos t-e^{t}\sin t & e^{t}\cos t+e^{t}\sin t
\end{array}
\right] = e^{3t}.
\]
Hence the inverse is
\begin{eqnarray*}
&&e^{-3t}\left[
\begin{array}{ccc}
e^{2t} & 0 & 0 \\
0 & e^{2t}\cos t+e^{2t}\sin t & -\left( e^{2t}\cos t-e^{2t}\sin \right) t \\
0 & -e^{2t}\sin t & e^{2t}\cos \left( t\right)
\end{array}
\right] ^{T} \\
&=& \left[
\begin{array}{ccc}
e^{-t} & 0 & 0 \\
0 & e^{-t}\left( \cos t+\sin t\right)  & -\left( \sin t\right) e^{-t} \\
0 & -e^{-t}\left( \cos t-\sin t\right)  & \left( \cos t\right) e^{-t}
\end{array}
\right]
\end{eqnarray*}
\end{expandable}
\end{problem}

\begin{problem}\label{prb:7.39} Find the inverse, if it exists, of the matrix
\begin{equation*}
A =
\left[
\begin{array}{ccc}
e^{t} & \cos t & \sin t \\
e^{t} & -\sin t & \cos t \\
e^{t} & -\cos t & -\sin t
\end{array}
\right]
\end{equation*}

Click the arrow to see answer.
\begin{expandable}
\begin{eqnarray*}
&&\left[
\begin{array}{ccc}
e^{t} & \cos t & \sin t \\
e^{t} & -\sin t & \cos t \\
e^{t} & -\cos t & -\sin t
\end{array}
\right] ^{-1} \\
&=&\left[
\begin{array}{ccc}
\frac{1}{2}e^{-t} & 0 & \frac{1}{2}e^{-t} \\
\frac{1}{2}\cos t+\frac{1}{2}\sin t & -\sin t & \frac{1}{2}\sin t-\frac{1}{2}
\cos t \\
\frac{1}{2}\sin t-\frac{1}{2}\cos t & \cos t & -\frac{1}{2}\cos t-\frac{1}{2}
\sin t
\end{array}
\right]
\end{eqnarray*}
\end{expandable}
\end{problem}

\begin{problem}\label{prb:7.40} Suppose $A$ is an upper triangular matrix. Show that $A^{-1}$ exists
if and only if all elements of the main diagonal are non zero. Is it true
that $A^{-1}$ will also be upper triangular? Explain. Could the same be concluded for lower triangular matrices?
\begin{hint}
The given condition is what it takes for the
determinant to be non zero. Recall that the determinant of an upper
triangular matrix is just the product of the entries on the main diagonal.
\end{hint}
\end{problem}

\begin{problem}\label{prb:7.41} If $A,B,$ and $C$ are each $n\times n$ matrices and $ABC$ is
invertible, show why each of $A,B,$ and $C$ are invertible.

Click the arrow to see answer.
\begin{expandable}
This follows
because $\det \left( ABC\right) =\det \left( A\right) \det \left( B\right)
\det \left( C\right) $ and if this product is nonzero, then each determinant
in the product is nonzero and so each of these matrices is invertible.
\end{expandable}
\end{problem}

\section*{Practice Problem Source}
These problems come from Chapter 3 of Ken Kuttler's \href{https://open.umn.edu/opentextbooks/textbooks/a-first-course-in-linear-algebra-2017}{\it A First Course in Linear Algebra}. (CC-BY)

Ken Kuttler, {\it  A First Course in Linear Algebra}, Lyryx 2017, Open Edition, pp. 272--315.   

\end{document}